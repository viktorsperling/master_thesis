% \glsresetall
\chapter{Conclusion} % Main chapter title
\label{Chapter8}

\lhead{Chapter 8. \emph{Conclusion}}

This chapter will conclude and evaluate the results shown in chapter \ref{Chapter7}. For the evaluation of the results, the introduced KPIs will be used, either as examples or references.
Last a final recommendation is made, under which circumstances or for which use case the used technologies should best used.
 
\section{Evaluation of the landscapes}

This section will evaluate the results for each implemented Node and compare it to its respective counterpart. This section is split into two subsection, one for each prototype.

\subsection{NPM/CDN prototype}

The Angular and Vue Nodes implemented were the following:

\begin{itemize}[noitemsep]
	\item Angular NPM 1
	\item Angular NPM 2
	\item Angular Unpkg 1
	\item Angular Unpkg 2
	\item Vue NPM 1
	\item Vue NPM 2
	\item Vue Unpkg 1
	\item Vue Unpkg 2
\end{itemize}

The naming refers to the technology used to load or access the required resources for the apps. Unpkg.com, as mentioned before, is a public cloud CDN. The idea behind those implementations is to prove that not only the CDN technology affects the performance of a micro frontend, but also the used frameworks.
Comparing the numbers of those two landscapes it becomes obvious that, the Angular landscapes load more resources (counted by the URLs loaded KPI) in a shorter loading time for all implementations. 
Also looking at the initial loading times, calculated through the loading procedure where caching was disabled, for all implementations Angular performed better compared to Vue.

It has to be mentioned, that the actual goal of this implementation was to compare the technology of the CDN, not the used frameworks. It still shows, that the framework has an affect on the performance of a micro frontend. 

Following conclusions can be drawn by comparing the numbers of the NPM implementations and the ones with Unpkg.com in use:

\begin{enumerate}
	\item The amount of requested URLs is significantly increased when CDN is used.
	\item Even though approximately ten times the amount of URLs is requested by the CDN landscapes, the initial loading time decreases by ~60 MS for the Angular Unpkg landscapes and for Vue it's even more.
	\item The cache usage of the CDN landscapes significantly increases. This information is taken from the amount of \textit{none established} ID occurrences. For each connection not established, a resource is loaded from the cache, thus the faster loading times, when cache is enabled.
	\item A similar result is present in the loaded bytes from connection type. Compared to the NPM implementations, the Unpkg environments load significantly more bytes from \textit{disk} or \textit{memory}. This behavior was anticipated in the CDN implementations, since it is one of the desired features of this technologies. Every resource has a designated URL where it is loaded from, thus the browser can distinguish if it already present or not.
	\item Additionally the average loaded content size of the Unpkg apps is lower, since only specific resources are requested from the CDN. Therefore big bundles with unused features are generally not present.
	\item The graph \ref{fig:unsed_imported_1} shows, that approximately half of the imported bytes were not used according to the Lighthouse report, the absolute amount of loaded bytes is significantly lower, compared with the NPM environments.
	\item When comparing the variances of the loading times, the Unpkg implementations show a lower value. Reasons for that could either be the protocol used by the CDN server or the smaller sizes of the loaded resources.
\end{enumerate}

In retrospective, the KPIs \textbf{loading time in MS, resource sizes in bytes} and \textbf{amount of cached resources} the Unpkg.com implementations follow the expected patterns. Additionally looking at the variance, the Unpkg environments show a less variant loading time for all applications compared with the NPM implementations. 
One behavior was not expected though, it was assumed that the initial loading time of a CDN landscape would be significantly higher. Reason for that assumption was the effect of the network latency, since the resources are loaded from a remote server, and not from an integrated bundle inside the project itself. Nonetheless, in case of the prototype this behavior could not be confirmed, as even the initial loading times for the Unpkg apps were lower compared to the NPM apps. Reason for that result could be the efficiently picked resources. Instead of importing whole bundles of libraries only necessary components or resources were added as imports in the Unpkg Nodes, thus the loaded byte size of those are so low.

Mentioning the \textit{"efficiently picked resources"} another KPI which should be considered for all landscapes, is the effort connected to using a corresponding technology. This metric is hard to measure though, since it is highly influenced by the individual using the technology. A developer who is familiar with Webpack for instance, would have less trouble using and implementing the Module Federation. Therefore this metric is not easily quantifiable. Still, in the context of this thesis, the author will try and provide a subjective opinion on his experiences with the implementations he has done as generally as possible. 
In case of the CDN the effort of implementation was comparably low. From a developer's point of view it's even less effort, since no libraries have to be maintained in a central resource or package manager file (namely \texttt{package.json}). Nonetheless it has to be considered what type of CDN is used.
In case of the prototype implementation, it was a public cloud CDN which already had all the required resources available. For a self-hosted CDN, this might not always be the case. Also when deciding to host an own CDN maintenance, development and deployment costs have to be put into account. This part was explained in chapter \ref{Chapter2}.
In summary, the CDN technology is a comparably easy way to avoid redundancies in a micro frontend landscape. It still is connected to certain obstacles, when the use case is highly specific and requires certain customizations on CDN side. 

Also multi-version landscapes are not supported by a CDN. That means that redundancies still can occur if the same resource is imported under different version tags. This use case is not directly covered by a CDN. If the resource itself has some sort of scoping feature a support can be provided (e.g. Custom Scopes by SAPUI5 Web Components). But this is not part of the CDN. Other technologies offer more support on that part.

\subsection{Web Component/WMF landscapes}

The functionality of the Web Component and Module Federation landscapes to avoid redundant libraries were explained in chapters \ref{Chapter4} and \ref{Chapter5}. Therefore, this section will focus on the direct comparison of these two environments. Since those two landscapes also include the aspect of heterogenic, multi-version micro frontends, this is considered in the comparison. 
Starting with values introduced in chapter \ref{Chapter6}, the first thing to attract attention should be the difference in the average loading times of the landscapes. For the caching disabled scenarios, the WMF landscape requires almost a fifth of the time compared to the compound landscapes. With caching enabled the difference is only a third but still significant. Additionally the bytes loaded by the landscapes differ. WMF loads almost double the amount of bytes compared to the compound environments in a shorter period on average. Additionally WMF does this with barely using the cache, even with caching enabled. Where the compound environment loaded approximately 756055 bytes on average from memory, the WMF only imported 1959 bytes on average from the cache. Therefore, based in this comparison the Webpack Module Federation seems to be on top.

Still further aspects have to be considered when using those technologies. The first is the multi-version handling. Web Components offer means to register different versions of a component, by adding e.g. a suffix to its tag name. In case of the prototype a component library provided such a feature. Still a self-developed component can implement a similar functionality. This again would be connected to more effort developing Web Components for a compound landscape in Luigi, but it is not entirely impossible.
When using an existing component library, this feature might already be present as it is the case for the SAPUI5 Web Components. 
Since each other version of a component is registered under a different tag, with a version suffix attached to it, the result might lead to redundancies again. That means a component called e.g. \texttt{ui5-table} would be registered and imported twice into the same landscape under different names. For instance version 1 would be \texttt{ui5-table-v1} and version 2 \texttt{ui5-table-v2}. Thus even though the different versions are handled in a distinguishable way, the redundancies would increase.

The Module Federation offers different means for solving this issue. By sharing certain dependencies and with a definition of a required version, redundancies are not entirely eliminated in a multi-version WMF landscape, but handled more elegantly. During the data collection of this landscape, a phenomenon appeared which seems to be intended by the Module Federation's developers. Shared dependencies like for instance \texttt{@angular-common - v1.1.2}, are always loaded via the network even if caching is enabled, but never in full size. On initial loading this dependency would be approximately 1.2 MB in size, but when reloading the page with cache enabled, it is loaded again but with 412 KB in size.
This behavior appeared for all shared dependencies in the WMF landscape. It is enabled via the Module Federation itself, since the shared dependencies are lazy loaded as chunks.\cite{wmf_the_good_and_ugly}

Again it seems to be that the WMF is superior on that regard compared to Web Components. Still following aspect is not considered yet, the effort of using the technologies. As mentioned in the previous section this is no actual KPI but rather a subjective opinion of the author. When directly compared, the effort of implementing or developing micro frontends, was by far higher when using WMF compared to Web Components. Even taking into account that a component library was used, still the Module Federation required more expertise with the Webpack bundler.
The possibilities with WMF are versatile, but to a layman unfamiliar with the necessary bundler Webpack, hard to use. The documentation offers good hints and explanations for options and syntax inside the configuration, but it is documented in a more general way and if the developer wants to use it in combination with a Webpack based framework like Angular or Vue, the whole operation becomes more experimental. In the case of Angular, a separate dependency is required to publish the hidden \texttt{webpack.config.js} via which the Module Federation is enabled in the first place. And this configuration has to be done for every module or remote, federated in the landscape. Applied to a real life scenario this might become a bigger obstacle, as it was in case of the prototype. Independent, isolated teams would work on the same micro frontend landscape, using different UI frameworks for their remote modules. Not only the bundle sizes of the heterogenic landscape would increase, since every shared dependency has to be bundled and published in the landscape, but the routing inside the landscape becomes a challenge itself. In case of Angular for instance, the inner routers do not recognize route changes. One router would have to import another application's router manually, in order to be able to communicate route changes.\cite{wmf_the_good_and_ugly}

Web Components on the other hand are based on standards. The result and assigning process is the same. Different UI frameworks offer means to register developed components as Web Components (e.g. Angular Elements). Thus the developer can work framework-independent within a familiar environment and expect the same result as another developer working with the tools of his choice.
Additionally Web Components are not bound to certain technology stacks - unlike the WMF which requires the Webpack bundler to enable it. Therefore, Web Components have less limitations and more stability due to their standardization.

In summary, both technologies offer means to solve the issue of avoiding redundant libraries, but when it comes to handle multi-version landscapes, WMF offers more elegant ways compared to Web Components. On the other hand using the WMF limits the developer to a certain technology stack and is not always easy or effortless implemented depending on the UI framework in use. Also certain obstacles are present in the WMF as of now and require workarounds to solve them.\cite{wmf_the_good_and_ugly} On that regard Web Components offer easier means and ways for implementation due to their standardized aspects.

\section{Final recommendations}

The previous sections contained general summaries for each respective technology, based on either empiric data, official sources or subjective experience. The following listing will provide final recommendations when and how best to use the introduced and implemented technologies, based on the conclusions made in this transcript.

\begin{itemize}
	\item \textbf{Content Delivery Networks} - CDN offer an easy way to reduce redundancies in micro frontend landscapes by centralizing the landscapes resources to one point. From a developer-perspective only the way of importing the resources changes. Therefore, this is the easiest and fastest way of achieving the goal of avoiding redundant libraries in micro frontend landscapes. Nonetheless, what it offers in simplicity it lacks in flexibility. Multi-version support is not always present and can not be entirely solved by this technology. Also a public CDN might not always have the necessary resources required by the landscape and hosting an own CDN is connected to high maintenance and developing costs (depending on its size). Therefore, if the use case describes a homogenic micro frontend landscape, one without different versions, CDN is the way to go. If different versions of the same resource are required, there are more elegant ways than this technology.
	
	\item \textbf{Web Components} - As a web standard it is a safe way of avoiding redundancies, maybe not in libraries but rather in the used components themselves. By providing reusable components to the browser which are not affected by the isolation of micro frontends, this technology reduces the amount of used components in the landscape. Additionally existing Web Component libraries offer means to scope different versions of its components, providing a way to handle multi-versions inside the landscape. With its standardized aspect it also does not limit the developer to certain technology stacks and is compatible with most common UI frameworks. Some even offer framework features to create Web Components from their projects like Angular Elements. If the use case requires a lot of reusable components with as less redundancies as possible, Web Components offer the best way to provide that service.
	
	\item \textbf{Webpack Module Federation} - A rather new technology compared to the other two, promising to excel where the other two lack. It can federalize any piece of precompiled code and serve it to the landscape. This module or remote is then embedded into a host application. The remote itself can either be a UI component, a module or a utility service, thus providing a flexible way of sharing code inside its landscapes. The possibilities are vast using this technology. But this offer comes for a price: One has to use the Webpack bundler and for certain features or issues workarounds are required. 
	Subsequently it can be said, this technology, if used correctly, is applicable to almost any use case as long as it involves a homogenic landscape \footnote{This was explained in section \ref{wmf_implementation_prototype} of chapter \ref{Chapter6}}.
\end{itemize}

As a closing word for the recommendation: When the main goal is to avoid redundant libraries in a micro frontend landscape, each of the introduced technologies offer means to do that. But each comes with its own kind of trade-off. Also when picking from one of the above choices, side effects and benefits have to be considered. Therefore, the final recommendation is highly dependent on the use case and requirements for the landscape to be developed.
Additionally it is not excluded or impossible to combine those technologies. 

