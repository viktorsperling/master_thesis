% !TeX document-id = {75b9b59a-7629-4c4a-9bf7-6c42d160f0fe}
%%%%%%%%%%%%%%%%%%%%%%%%%%%%%%%%%%%%%%%%%
% Masters/Doctoral Thesis
% LaTeX Template
% Version 1.43 (17/5/14)
%
% This template has been downloaded from:
% http://www.LaTeXTemplates.com
%
% Original authors:
% Steven Gunn
% http://users.ecs.soton.ac.uk/srg/softwaretools/document/templates/
% and
% Sunil Patel
% http://www.sunilpatel.co.uk/thesis-template/
%
% License:
% CC BY-NC-SA 3.0 (http://creativecommons.org/licenses/by-nc-sa/3.0/)
%
% Note:
% Make sure to edit document variables in the Thesis.cls file
%
%%%%%%%%%%%%%%%%%%%%%%%%%%%%%%%%%%%%%%%%%

%----------------------------------------------------------------------------------------
%	PACKAGES AND OTHER DOCUMENT CONFIGURATIONS
%----------------------------------------------------------------------------------------
\documentclass[
english,
oneside,
paper=A4,
fontsize=11pt,
BCOR=2mm,				% Bindekorrektur: 2mm 
DIV=default,			% Seitenaufteilung an A4 anpassen
open=any,				% Kapitel-Anfang links oder rechts
listof=toc, 
bibliography=totoc,
parskip=half			% Abstand zwischen Absätzen
]{Thesis} % The default font size and one-sided printing (no margin offsets)

\usepackage{svg}

\usepackage{listings}

\usepackage{multirow}

\usepackage{graphicx}

\usepackage{xltabular}

\usepackage{pdfpages}

\usepackage{titlesec}

\usepackage{enumerate}

\usepackage{enumitem}

\usepackage{mathtools}

\usepackage{makecell}

\usepackage{longtable}

\usepackage{lscape}

\usepackage{adjustbox}

\usepackage{booktabs}

\usepackage[english,ngerman]{babel}

\usepackage{color}

\usepackage{upquote}

\usepackage{listings}

\usepackage{hyperref}
\def\UrlBreaks{\do\/\do-}

\usepackage{float}

\usepackage{aliascnt}

\newaliascnt{eqfloat}{equation}
\newfloat{eqfloat}{h}{eqflts}
\floatname{eqfloat}{Equation}

\newlength{\mycolwidth}
\newlength{\mycolwidthtwo}

\newcommand*{\ORGeqfloat}{}
\let\ORGeqfloat\eqfloat
\def\eqfloat{%
	\let\ORIGINALcaption\caption
	\def\caption{%
		\addtocounter{equation}{-1}%
		\ORIGINALcaption
	}%
	\ORGeqfloat
}

\definecolor{lightgray}{rgb}{0.95, 0.95, 0.95}
\definecolor{darkgray}{rgb}{0.4, 0.4, 0.4}
\definecolor{purple}{rgb}{0.65, 0.12, 0.82}
\definecolor{editorGray}{rgb}{0.95, 0.95, 0.95}
\definecolor{editorOcher}{rgb}{1, 0.5, 0} % #FF7F00 -> rgb(239, 169, 0)
\definecolor{editorGreen}{rgb}{0, 0.5, 0} % #007C00 -> rgb(0, 124, 0)
\definecolor{orange}{rgb}{1,0.45,0.13}		
\definecolor{olive}{rgb}{0.17,0.59,0.20}
\definecolor{brown}{rgb}{0.69,0.31,0.31}
\definecolor{purple}{rgb}{0.38,0.18,0.81}
\definecolor{lightblue}{rgb}{0.1,0.57,0.7}
\definecolor{lightred}{rgb}{1,0.4,0.5}


\DeclarePairedDelimiter\round{\lfloor}{\rceil}

%CSS
\lstdefinelanguage{CSS}{
	keywords={color,background-image:,margin,padding,font,weight,display,position,top,left,right,bottom,list,style,border,size,white,space,min,width, transition:, transform:, transition-property, transition-duration, transition-timing-function},	
	sensitive=true,
	morecomment=[l]{//},
	morecomment=[s]{/*}{*/},
	morestring=[b]',
	morestring=[b]",
	alsoletter={:},
	alsodigit={-}
}

% JavaScript
\lstdefinelanguage{JavaScript}{
	morekeywords={typeof, new, true, false, catch, function, return, null, catch, switch, var, if, in, while, do, else, case, break,class,super,let,const,this,constructor,import,from,export,default,extends,render,for,of,@Component,@Input,implements,private,@ViewChild,await,type,string},
	morecomment=[s]{/*}{*/},
	morecomment=[l]//,
	morestring=[b]",
	morestring=[b]',
	morestring=[b]`,
}

\lstdefinelanguage{HTML5}{
	language=html,
	sensitive=true,	
	alsoletter={<>=-},	
	morecomment=[s]{<!-}{-->},
	tag=[s],
	otherkeywords={
		% General
		>,
		% Standard tags
		<!DOCTYPE,
		</html, <html, <head, <title, </title, <style, </style, <link, </head, <meta, />, <ul, <li,</li, </ul, <slot,</slot,<template, </template, </my-paragraph, <my-paragraph, </custom-element, <custom-element, <awesome-explosion, </awesome-explosion,<ui5-table-placeholder,<span,</span,</ui5-table-column-placeholder,<ui5-table-column-placeholder,</ui5-table-placeholder,
		% body
		</body, <body,
		% Divs
		</div, <div, </div>, 
		% Paragraphs
		</p, <p, </p>,
		% scripts
		</script, <script,<h2,</h2,
		% More tags...
		<canvas, /canvas>, <svg, <rect, <animateTransform, </rect>, </svg>, <video, <source, <iframe, </iframe>, </video>, <image, </image>, <header, </header, <article, </article
	},
	ndkeywords={
		% General
		=,
		% HTML attributes
		charset=, src=, id=, width=, height=, style=, type=, rel=, href=,
		% SVG attributes
		fill=, attributeName=, begin=, dur=, from=, to=, poster=, controls=, x=, y=, repeatCount=, xlink:href=,
		% properties
		margin:, padding:, background-image:, border:, top:, left:, position:, width:, height:, margin-top:, margin-bottom:, font-size:, line-height:,
		% CSS3 properties
		transform:, -moz-transform:, -webkit-transform:,
		animation:, -webkit-animation:,
		transition:,  transition-duration:, transition-property:, transition-timing-function:,
	}
}

\lstdefinestyle{htmlcssjs} {%
	% General design
	%  backgroundcolor=\color{editorGray},
	basicstyle={\footnotesize\ttfamily},   
	frame=b,
	% line-numbers
	xleftmargin={0.75cm},
	numbers=left,
	stepnumber=1,
	firstnumber=1,
	numberfirstline=true,	
	% Code design
	identifierstyle=\color{black},
	keywordstyle=\color{blue}\bfseries,
	ndkeywordstyle=\color{editorGreen}\bfseries,
	stringstyle=\color{editorOcher}\ttfamily,
	commentstyle=\color{brown}\ttfamily,
	% Code
	language=HTML5,
	alsolanguage=JavaScript,
	alsodigit={.:;},	
	tabsize=2,
	showtabs=false,
	showspaces=false,
	showstringspaces=false,
	extendedchars=true,
	breaklines=true,
	% German umlauts
	literate=%
	{Ö}{{\"O}}1
	{Ä}{{\"A}}1
	{Ü}{{\"U}}1
	{ß}{{\ss}}1
	{ü}{{\"u}}1
	{ä}{{\"a}}1
	{ö}{{\"o}}1
}

%Config for YAML files
\newcommand\YAMLcolonstyle{\color{red}\mdseries}
\newcommand\YAMLkeystyle{\color{black}\bfseries}
\newcommand\YAMLvaluestyle{\color{blue}\mdseries}

\makeatletter

% here is a macro expanding to the name of the language
% (handy if you decide to change it further down the road)
\newcommand\language@yaml{yaml}

\expandafter\expandafter\expandafter\lstdefinelanguage
\expandafter{\language@yaml}
{
	keywords={true,false,null,y,n},
	keywordstyle=\color{darkgray}\ttfamily,                               % assuming a key comes first
	sensitive=false,
	comment=[l]{\#},
	morecomment=[s]{/*}{*/},
	commentstyle=\color{purple}\ttfamily,
	stringstyle=\color{green}\ttfamily,
	tabsize=2,
	moredelim=[l][\color{orange}]{\&},
	moredelim=[l][\color{magenta}]{*},
	moredelim=**[il][\YAMLcolonstyle{:}\YAMLvaluestyle]{:},   % switch to value style at :
	morestring=[b]',
	morestring=[b]",
	literate =    {---}{{\ProcessThreeDashes}}3
	{>}{{\textcolor{red}\textgreater}}1     
	{|}{{\textcolor{red}\textbar}}1 
	{\ -\ }{{\mdseries\ -\ }}3,
}

% switch to key style at EOL
\lst@AddToHook{EveryLine}{\ifx\lst@language\language@yaml\YAMLkeystyle\fi}
\makeatother

\newcommand\ProcessThreeDashes{\llap{\color{cyan}\mdseries-{-}-}}

\lstset{%
	% Basic design
	backgroundcolor=\color{editorGray},
	basicstyle={\small\ttfamily},   
	frame=l,
	% Line numbers
	numbers=left,
	stepnumber=1,
	firstnumber=1,
	numberfirstline=true,
	% Code design   
	keywordstyle=\color{blue}\bfseries,
	commentstyle=\color{darkgray}\ttfamily,
	ndkeywordstyle=\color{editorGreen}\bfseries,
	stringstyle=\color{editorOcher},
	% Code
	language=HTML5,
	alsolanguage=JavaScript,
	alsodigit={.:;},
	tabsize=2,
	showtabs=false,
	showspaces=false,
	showstringspaces=false,
	extendedchars=true,
	breaklines=true,        
	% Support for German umlauts
	literate=%
	{Ö}{{\"O}}1
	{Ä}{{\"A}}1
	{Ü}{{\"U}}1
	{ß}{{\ss}}1
	{ü}{{\"u}}1
	{ä}{{\"a}}1
	{ö}{{\"o}}1
}

\titleformat{\chapter}[block]
{\normalfont\huge\bfseries}{\thechapter.}{1em}{\Huge}
\titlespacing*{\chapter}{0pt}{-19pt}{0pt}

\graphicspath{{Pictures/}} % Specifies the directory where pictures are stored

%\usepackage[square, numbers, comma, sort&compress]{natbib} % Use the natbib reference package - read up on this to edit the reference style; if you want text (e.g. Smith et al., 2012) for the in-text references (instead of numbers), remove 'numbers'
%\usepackage{cite}
\usepackage{float}
\hypersetup{urlcolor=black, colorlinks=true}
\hypersetup{
	colorlinks=true,
	%allcolors=red,
	anchorcolor=red,
	linkcolor=red,          % color of internal links (change box color with linkbordercolor)
	citecolor=green,        % color of links to bibliography
	filecolor=magenta,      % color of file links
	runcolor=red,		    
	urlcolor=cyan} 	        % Colors hyperlinks in blue - change to black if annoying
\title{\ttitle} % Defines the thesis title - don't touch this

\begin{document}
	
	\frontmatter % Use roman page numbering style (i, ii, iii, iv...) for the pre-content pages
	
	\setstretch{1.3} % Line spacing of 1.3
	
	% Define the page headers using the FancyHdr package and set up for one-sided printing
	\fancyhead{} % Clears all page headers and footers
	\rhead{\thepage} % Sets the right side header to show the page number
	\lhead{} % Clears the left side page header
	
	\pagestyle{fancy} % Finally, use the "fancy" page style to implement the FancyHdr headers
	
	\newcommand{\HRule}{\rule{\linewidth}{0.5mm}} % New command to make the lines in the title page
	
	\newcommand{\brparagraph}[1]{\paragraph{#1}\mbox{}\\}
	
	\newcommand{\source}[1]{\caption*{Source: {#1}} }
	
	% PDF meta-data
	\hypersetup{pdftitle={\ttitle}}
	\hypersetup{pdfsubject=\subjectname}
	\hypersetup{pdfauthor=\authornames}
	\hypersetup{pdfkeywords=\keywordnames}
	
	%----------------------------------------------------------------------------------------
	%	TITLE PAGE
	%----------------------------------------------------------------------------------------
	
	
	\includepdf[offset=32mm -30mm]{Figures/Thesisdeckblatt.pdf}
	\newpage
	
	\section*{Eidesstattliche Erklärung}
	
	Ich erkläre an Eides statt, dass ich die hier vorgelegte Master-Thesis selbstständig und
	ausschließlich unter Verwendung der angegebenen Literatur und sonstigen Hilfsmittel verfasst habe.
	Die Arbeit wurde in gleicher oder ähnlicher Form keiner anderen Prüfungsbehörde zur Erlangung
	eines akademischen Grades vorgelegt.
	
	\vspace{1.3cm}
	
	\noindent\rule[-0.1cm]{6cm}{0.5pt}\\
	Karlsruhe, den \today
	\newpage
	%----------------------------------------------------------------------------------------
	%	Confidentiality Agreement
	%----------------------------------------------------------------------------------------
	\section*{Master Thesis Confidentiality Agreement}	
	Not required yet.
%	Title of thesis:  \textit{Effort analysis of application porting to Kubernetes based on a prototype}
	
%	Name of student: 
%	\textit{Viktor Sperling}
%	Registration no. at university: \textit{48370}, 
%	Course: \textit{Wirtschaftsinformatik}
%	The student is writing a thesis in cooperation with SAP SE (herein: SAP).
%	\textbf{Hochschule Karlsruhe für Technik und Wirtschaft}, represented by its president, and the thesis examiners
	
%	1. \textit{Prof. Dr. Udo Müller} (first examiner),
	
%	2. \textit{Prof. Dr. Rainer Neumann} (second examiner),
	
%	herein: Examiners
%	acknowledge that the thesis and the confidential information about SAP that it contains are trade secrets.

%	The University and the Examiners undertake to treat all unpublished company information, including without limitation technical and business information, plans, experience, knowhow, designs and other documents to which they have access in connection with the thesis as confidential and will not make it available to third parties.        
	
%	The abovementioned duty of confidentiality does not extend to information:
%	1.	That they already knew before this agreement entered force
%	2.	That they rightfully receive from third parties and that is not subject to a duty of confidentiality
%	3.	That is or becomes public knowledge without breach of confidentiality under this agreement
%	4.	That they develop independently
	
%	If university or examination regulations require other examiners to assess the thesis (for example, members of the standing examination committee or university directors), SAP must be informed of their names and they must join this agreement. 
%	The University and the Examiners must mark this thesis as confidential and must not let anyone gain access to it. They must not publish or reproduce the thesis in whole or in part without SAP's express permission. 
%	SAP can bring claims against the University and the Examiners on grounds of a breach of this agreement only if SAP can show their intent or gross negligence.
%	Only German law applies. 

%	First examiner:\hspace{5.4cm}Second examiner:
	
%	\noindent\rule[-0.1cm]{3cm}{0.5pt}\hspace{5cm}\noindent\rule[-0.1cm]{3cm}{0.5pt}\\                      		
%	(Name: Prof. Dr. Udo Müller)\hspace{2.9cm}(Name: Prof. Dr. Rainer Neumann)
		
%	Place: Karlsruhe, den\hspace{4.3cm}				Place: Karlsruhe, den
	
%   Date: \noindent\rule[-0.1cm]{3cm}{0.5pt}\hspace{4cm}Date: \noindent\rule[-0.1cm]{3cm}{0.5pt}
	
	
%	Title of thesis: \textit{Effort analysis of application porting to Kubernetes based on a prototype}
	
%	Name of student: \textit{Viktor Sperling}
	
%	The student is writing a thesis in cooperation with SAP SE. This present thesis on 
	
%	\textit{Effort analysis application porting to Kubernetes based on a prototype}
	
%	contains trade secrets of SAP SE.
%	This thesis has been identified as confidential. Unauthorized persons and organizations are not granted access to this thesis. With the exception of the thesis supervisor, second examiner, and the examining committee of Hochschule Karlsruhe für Technik und Wirtschaft university, no third party is granted access to the thesis or any part of it. The thesis must not be published or reproduced in whole or in part.
	
%	The thesis must not be published, reproduced, or transmitted in whole or in part without the express permission of SAP SE. Only the examiners and the members of the examining committee of Hochschule Karlsruhe für Technik und Wirtschaft university are granted access to this thesis.
%	The confidentiality of the thesis ends no earlier than three months after the publication  and no later than five years after the examination processes are completed.
	
%	\noindent\rule[-0.1cm]{6cm}{0.5pt}\\
%	(Student’s Name: Viktor Sperling)

%	Place: \noindent\rule[-0.1cm]{4cm}{0.5pt} \hspace{3cm} Date:\noindent\rule[-0.1cm]{4cm}{0.5pt}\\
	\newpage

	
	%----------------------------------------------------------------------------------------
	%	QUOTATION PAGE
	%----------------------------------------------------------------------------------------
	
	%\pagestyle{empty} % No headers or footers for the following pages
	
	%\null\vfill % Add some space to move the quote down the page a bit
	
	%\textit{``Some deep quote \textbf{TO DO}"}
	
	%\begin{flushright}
	%	Add a quote here
	%\end{flushright}
	
	%\vfill\vfill\vfill\vfill\vfill\vfill\null % Add some space at the bottom to position the quote just right
	
	%\clearpage % Start a new page
	
	%----------------------------------------------------------------------------------------
	%	ABSTRACT PAGE
	%----------------------------------------------------------------------------------------
	
	\addtotoc{Abstract} % Add the "Abstract" page entry to the Contents
	
	% \selectlanguage{ngerman}
	% \abstract{ % Add a gap in the Contents, for aesthetics
	
	% Abstrakt Deutsch.
	
	% Mit Absätzen arbeiten.
	
	% }
	
	% \clearpage % Start a new page
	
	\selectlanguage{english}
	\abstract{ % Add a gap in the Contents, for aesthetics
		
		This is an amazing Abstract!
		
%		In today's development projects, the runtime environment has to be chosen wisely.
%		Depending on the landscape the application is deployed in, its behaviour changes drastically.
%		And since the trend itself goes to developing microservices, the number of possible runtime environments increases.
%		The decision, to ensure a proper compromise between development effort and cost efficiency, depends on various factors.
		
%		One way of approaching this issue is to consider the existing landscape offered by the company. 
%		Some companies have their own processes and best practices for their products and offer a proper landscape and support for them. 
%		Then again, this does not necessarily mean that this solution is the best one.
		
%		The best approach can also be found via market analysis. That way, common design patterns on the open market can be acquired and established in the company, with the benefit of gaining an educational value for the development team participating in such a project. 
%		With this approach, one cannot help but find that Kubernetes has established a significant presence on the market and should be considered as the deployment and runtime environment for a project.
		
%		This thesis will attempt to prove the value an application can gain by being deployed into Kubernetes and show when it makes sense to use this tool and when it does not.
		
%		For this purpose, a prototype Kubernetes cluster was created, on which then specific tests were applied.
%		To demonstrate the benefits the applications gains inside this landscape, a regular setup with VM based deployments was created as well.
%		The specific tests were applied to this setup, too.
		
%		Via this approach, a data basis was created on which the proof is based.
%		The data gathered from the tests was later visualized with a python script.
		
%		For the conclusion, not only the test results were taken into account, but also the applied effort of deploying and creating those setups.
%		It has to be considered, however that the goal was not to compare the applications but rather the environments.
		
%		The gathered results can be used for further analysis or be used in Kubernetes based projects, by applying certain rules and best-practices.		
	}
	
	\clearpage % Start a new page
	
	%----------------------------------------------------------------------------------------
	%	LIST OF CONTENTS/FIGURES/TABLES PAGES
	%----------------------------------------------------------------------------------------
	
	\pagestyle{fancy} % The page style headers have been "empty" all this time, now use the "fancy" headers as defined before to bring them back
	
	\lhead{\emph{Content}} % Set the left side page header to "Contents"
	\setcounter{tocdepth}{3}
	\tableofcontents % Write out the Table of Contents
	
	%----------------------------------------------------------------------------------------
	%	THESIS CONTENT - CHAPTERS
	%----------------------------------------------------------------------------------------
	
	\mainmatter % Begin numeric (1,2,3...) page numbering
	
	\pagestyle{fancy} % Return the page headers back to the "fancy" style
	
	% Include the chapters of the thesis as separate files from the Chapters folder
	% Uncomment the lines as you write the chapters
	
	%include chapters here like this:
	% \glsresetall
\chapter{Introduction} % Main chapter title
\label{Chapter1}

\lhead{Chapter 1. \emph{Introduction}}

This chapter will explain the problem, motivation and goals of this transcript. 
It will also narrow down the thesis scope.

\section{Motivation}

The micro frontend technology has experienced a rise in popularity over the past 3 years. The occurrence of such a trend is no surprise, considering the fact that more and more applications are designed following the micro service architecture principles.\cite{google_micro_frontend_trends} 
Micro frontends share a common interest with their assumed precursor. Generally it can be stated, that the market prefers smaller independent lightweight applications over monolithic giants with strong dependencies. Similar to micro services, micro frontends offer almost the same advantages to a development team: Independence of the respective micro frontend teams, shorter and decoupled developments cycles, smaller scope of maintenance, separations of concerns, reusable components available via defined APIs and further more.\cite{advantages_of_mfes} But still this concept has certain flaws which have to be considered when making the decision.\cite{Yang_2019}

Depending on the type of the to-be-developed application, the concept of micro frontends is not always favored. One major point of criticism is the aspect of redundancies inside the landscape. Specifically the aspect of isolation is a core aspect of this architecture. The circumstance that, each micro frontend is a separate project with its own runtime, developed by an independent team causes the issue to occur. It allows the development teams to use their own preferred tech stack and enables the application itself to be reused in different landscapes. Nonetheless, that also means that isolated micro frontends are partially loading and using the same libraries during their runtime. This is causing an overhead on client and server side and therefore correlates with the runtime costs of such a landscape and potentially an inferior user experience.\cite{motivation_benefits_adopting_MFs}\cite{micro_frontends_in_general}

Despite isolation being a key aspect of this architecture, it is still possible to avoid the caused redundancies without disabling it.

\section{Goals}

The goal of this thesis is to provide proof and evaluate methods to avoid redundancies during the runtime of a representative micro frontend landscape. 

For the context of this transcript, the term \textit{"redundancies"} or \textit{"redundant libraries"} means same libraries or dependencies used by multiple micro frontend projects. It does not mean the redundant code elements, like reoccurring methods inside the projects or same CCS classes. 

A library is determined by its identifier or its name, e.g. \texttt{@luigi-project.} If the given library or dependency is imported, via any means and in any version, by multiple different micro frontends, it is considered to be redundant for the context of this thesis. The case of different versions of redundant libraries is considered too and is covered in the following chapters.

To achieve the mentioned goal, two prototypes were implemented using the Luigi micro frontend framework. The landscape itself was implemented to be as representative as possible by using a heterogenic tech stack. Additionally following methods were implemented to avoid redundantly loaded libraries:

\begin{itemize}
	\item Content Delivery Network (CDN)
	\item Web Components in Luigi Compound views
	\item Webpack Module Federation (WMF) framework
\end{itemize}

The implemented methods are then evaluated under predefined metrics. The steps followed to collect the data for the metrics are explained too.
Eventually a conclusion is made based on the empiric data, accompanied by the subjective experience of the author. It is explained how the usage and implementation of the respective technologies was experienced and which advantages each technology brings to the table from an objective and subjective view.
  
\section{Scope of Work}

Since the time given for this transcript is limited, the context is scoped. Therefore, following aspects are either not considered or referenced on a theoretical level.

\begin{itemize}
	\item The development of an own CDN - An approximate cost assumption is made for such a project, without actual implementation.
	\item Usage of the Webpack bundler outside the context of the Module Federation - Since this bundler offers many ways for configuration, the focus lies on the necessary onces for the WMF context.
	\item Configuration for the Luigi framework of the implemented landscape - Each prototype implemented uses this framework for their micro frontend landscapes and since not all configurations can be shown here, only a general overview is given.
	\item Development of Web Components - The general usage and functionality of this standard is explained.
	\item Combinations of the implemented technologies - This aspect is not empirically considered in this transcript.
\end{itemize}

\section{The collaborators}

This thesis was written in collaboration with SAP Luigi project team. The SAP originally founded in in 1972 as SAPD (system analysis program development), later became SAP AG in 2005 and in 2014 SAP SE. The Hybris company originally founded in 1997 was later acquired by the SAP in 2013, after moving its headquarter from Zug in Switzerland to Munich in Germany. The Luigi project team is part of the SAP Hybris organization but the developed framework is an open-source product. 
The team and their product aim to improve the experiences of customers, developers and administrators who are using Luigi. Their open-source product was designed to make the transformation from monolithic architectures into micro frontend based landscapes as smooth as possible. 

 

	% \glsresetall
\chapter{State of technology} % Main chapter title
\label{Chapter2}

\lhead{Chapter 2. \emph{State of technology}}

In this chapter the current state of technologies which can provide possible solutions for the issue of redundant resources in a micro frontend landscape is explained. Additionally, an overview about the used Luigi framework is given.

\section{Micro frontend framework Luigi}
\label{mf_framework_luigi}

Currently there are different micro frontend frameworks present on the market. Some of which are \textbf{Bit, SystemJS, Webpack 5 Module Federation, Piral, Single SPA} and \textbf{Luigi}.\cite{top10_mffs}
For the implementation of the representative landscape the Luigi framework was used, therefore a short introduction of it is given here.

\subsection{Luigi}

Luigi is an open source JavaScript framework for micro frontends, consisting of two main parts, the \textbf{Luigi Core} and the \textbf{Luigi Client}. It provides a basis to integrate micro frontends, also called \textbf{Nodes} in this context. Both parts serve different purposes in the micro frontend landscape.\cite{luigi_doc_overview}

\subsubsection{Core}

The \textbf{Luigi Core} is the groundstone of the whole landscape. It defines the main app, which will serve as an entry point for the user. The possible, applicable configurations of this part are the following:

\begin{itemize}[noitemsep]
	\item Navigation
	\item Authorization
	\item Localization - providing translations to display its applications in multiple languages
	\item General settings - e.g. header display configurations, apply loading indicators for the micro frontends, etc.
	\item Luigi Core API - providing functions for the core app to interact with the framework and access its features
\end{itemize} 

The core application inside the landscape is determined via the \texttt{luigi.config.js}. Any project containing it can fulfill this role, which is the only prerequisite. In the context of the framework this app is later referred to as the \textbf{Core}.
Following this principle, the project structure of the \textbf{Core} could look as shown in \ref{luigi_core_structure}.\footnote{This example is directly taken from the implemented core project.}  

\begin{lstlisting}[language=Bash, caption=Project structure for a Luigi Core application including the \texttt{luigi.config.js}, label=luigi_core_structure]
	- react-core-mf
		- [...]
		- node_modules
		- public
			- [...]
			- index.html
			- luigi.config.js
		- [...]
		- src
\end{lstlisting}

In the \texttt{luigi.config.js} itself the above mentioned possible configurations are defined.\cite{luigi_doc_core}

\subsubsection{Client}

The Luigi Client serves as the connection between the framework and its micro frontends. In order to establish the connection a micro frontend has to import and initialize it. This will grant the micro frontend access to the \textbf{LuigiClient} object during the runtime. The micro frontend can then interact with the framework to e.g. set a global state, add event listeners or enable navigation between other micro frontends in the landscape.

An import of the Luigi Client can be accomplished via different methods. The most forward approach, might be the direct import with a HTML script tag. Another option can be the import of the local package manager dependency.\cite{luigi_client}

\begin{lstlisting}[caption=Import methods of the Luigi Client]
	<!-- Via the HTML script tag -->
	<script src="https://unpkg.com/@luigi-project/client@1.17.0/luigi-client.js">
	</script>
	
	<!-- Via the package manager -->
	import LuigiClient from "@luigi-project/client";
\end{lstlisting}

\subsection{Architecture}

After the short introduction of the framework's core components, a general architecture of a landscape implemented with it is provided below.

\begin{figure}[!h]
	\centering
	\includegraphics[width=1\textwidth]{Figures/Luigi_Architektur.png}
	\caption{Architecture of the Luigi Framework \cite{luigi_architecture}}
	\label{fig:luigi_architecture_fig}
\end{figure}

As can be seen in figure \ref{fig:luigi_architecture_fig}, the displayed micro frontends in the Web App have a distinguished position in the Micro Frontends Container. The embedded projects can either be integrated via iFrames or as Web Components as so called Compounds.
Via the provided dashboard, called Navigation Frame in the image, the Luigi Core features are used. Through it the navigation or search feature can be accessed.

As described in the first chapter of this transcript, the issue in such a landscape is redundancies between the same libraries of the embedded micro frontends. Each Node is an isolated application, which is the reason for this problem.
For instance, navigating to the first micro frontend and then followed by the second one will load the bundled projects with their corresponding dependencies. In this case the browser has no way to distinguish if a loaded resource is already present or not. Reason for that is the bundling of the projects which makes the references indistinguishable to the browser.
The sections below will introduce methods of resolving exactly that issue. 

\section{Content Delivery Networks}
\label{cdn_intro}

One way of avoiding loading redundant libraries into a micro frontend landscape can be a central source, where all resources are stored and can be requested via an API. Such a concept exists in the idea of Content Delivery Networks (CDNs).
A CDN is the simplest way of solving the issue at hand. Since the resources are loaded via the CDN API, a dependency is available under a unique URL. Loading a dependency twice means the browser requests the resource from the same URL with the same parameters. Therefore, the browser can determine if the given resource is already present and decide to either load it from the CDN or from the browser cache, if it is already loaded.

Nonetheless, a CDN has its limits which have to be considered. Costs can play a key role in the decision process.
Even though free CDN services like \textbf{Unpkg.com} or \textbf{cdnjs} are valid options, the advantages of using such might differ depending on the business case.
For instance, those services might solve the issue of redundant libraries in a micro frontend landscape, but do they really improve the user experience of the application? 
Since the resources would be loaded from a cloud based service, the server location has significant influence on the latency of the requests. Therefore, depending from where the website is accessed, this circumstance could lead to a performance decrease when initially loading the page.

Also the technological dependency on a public cloud based CDN has to be considered. Most cloud based services are scalable and replicable. However, a down time could lead to immense costs for a platform provider relying on a public CDN to deliver the platform resources.
Besides, it is never guaranteed that certain dependencies are always available on the public CDN.

Another option is self-hosting a CDN and providing all required resources independently. A business would have to acquire hardware-resources (physically or via a hardware-as-a-service provider like Amazon Web Services), maintain the servers, fill the CDN with the necessary content and keep it up to date. The costs attached to this scenario can be so immense and exclude this method due to cost inefficiency.\cite{Meassuring_a_commercial_CDN}

\section{Web Components}

Another option to address the issue of redundancies are Web Components. Consisting of the following four standards, they provide reusable elements closed in HTML tags.\cite{mdn_web_docs}

\begin{itemize}[noitemsep]
	\item Custom Elements
	\item Shadow DOM
	\item HTML Template
	\item ES Modules
\end{itemize}


The necessary feature to achieve the goal of avoiding the issue is mainly provided by the \textbf{Custom Elements} standard of Web Components, therefore a short explanation of its behavior is given here.

\subsection{Custom Elements}

This standard of Web Components provides an API via which new HTML tags can be defined and registered by the \texttt{CustomElementRegistry}. Since one tag name can only be registered once, multiple registrations of the same element would lead to an exception. Thus, making already registered tags reusable for the whole micro frontend landscape without reloading the code of the registered element.\cite{google_reusable_wcs}

\begin{figure}[!h]
	\centering
	\includegraphics[width=0.6\textwidth]{Figures/customElements_registered.drawio.png}
	\caption{Simple micro frontend landscape using Web Components}
	\label{fig:same_wc_example}
\end{figure}

Figure \ref{fig:same_wc_example} shows a simple micro frontend landscape with Luigi using Web Components. As it is visible, both micro frontends are registering the same tag name \texttt{custom-element} using the \texttt{customElements} object. This object is a read-only property of the \texttt{Window} interface.
The above shown scenario will cause a  \texttt{DOMException}, this way making it impossible to register the same tag twice.

The registration of Web Components follows the \textit{"first come, first served"} principle, the first micro frontend to register a tag name defines this tag and can not be overwritten without reloading the page.\cite{mdn_web_docs_define}

\subsection{Objections}

There is still a niche case which can be raised as an objection. What if even though the tag name is the same, the elements themselves are different? This is a justified objection to the example above. And a conclusion is, that it it is up to the developers to use the standard properly.

Assuming a common micro frontend landscape where every micro frontend is developed by independent, isolated teams, using heterogenic frameworks, the mentioned issue might occur. Two teams use Web Components and try to register different elements under the same tag name in the same landscape. This requires organizational interference on team level.

One way would be to define a name space for every team when creating Web Components e.g. Team 1 has the name space \texttt{team-1}. Making a Web Component registered by the micro frontend of Team 1, be named \texttt{team-1-tagname}. That again could lead to redundancies, because there is no way of assuring that the \texttt{team-1-tagname} and \texttt{team-2-tagname} Web Components are not the same.\cite{wc_best_practices}

A better way might be to assign a common Web Components library like \textbf{UI5 Web Components}. That way the registered elements are provided by the library and are limited to set of unique registrable tags. If another micro frontend would use an already registered element of the library, a warning is thrown but no error occurs.

Most Web Component libraries also offer so called scoping options. This feature finds its use case, when versions of the used components differ from micro frontend to micro frontend. This option enables the developers to customize their Web Components and register them under different tags. That of course might lead to redundancies again. But it also reduces the dependency of the developer teams to always use the latest version of the component library or the first registered element in the landscape. \cite{ui5_webcomponents_scoping} \cite{openwc_scoping}

\section{Webpack Module Federation}
\label{wmf_chap_2}

A rather new way of creating micro frontends is the \textbf{Webpack 5 Module Federation (WMF)}. With the release of version 5 in October 10 in 2020, this technology added certain features which improved its usage for developing micro frontends.\cite{wmf_concepts}

Since the main focus of this document is to propose methods of avoiding redundancies in micro frontend landscapes, an introduction of this WMF feature is given here.

The basic concept of the new WMF involves so called \texttt{hosts} and \texttt{remotes}. These terms are comparable to the \textbf{Luigi Core} and \textbf{Luigi Nodes} principle, previously introduced in section \ref{mf_framework_luigi}. In case of WMF the \texttt{host} aka \texttt{shell} represents the \textbf{Core} or basis of the micro frontend landscape and the \texttt{remotes} or \texttt{micro frontends} are the \textbf{Nodes} integrated or loaded into it. 
Still a direct comparison is not entirely possible since both frameworks look similar functionality-wise, but in their cores work differently. For instance, when using WMF one is tied to use the Webpack bundler, since the necessary configuration is done in the \texttt{webpack.config.js}. This restriction is applied to every micro frontend in the WMF micro frontend landscape, not only to the \texttt{host}.

WMF can be used in combination with most of the common UI Frameworks. But since the implementation for this thesis was done with Angular, the further examples and explanations will be Angular-based.

As explained WMF is a micro frontend framework, but for the purpose of solving the issue introduced in this transcript, it provides functionality too.
This feature is enabled and configured, as well as the rest of the WMF, via the \texttt{webpack.config.js}. When configuring the components of the landscape, it is possible to define a section where shared dependencies are described. These dependencies can be defined in different ways. For instance, it is possible to define a strict version of the dependency, which would result in the framework loading this specific version. Or one can define a less restricted dependency which would mean, that if another \texttt{remote} loads the same dependency but in a different version, the framework would automatically apply the highest major version of the dependency to both micro frontends.
\newpage
\begin{lstlisting}[language=JavaScript, caption=Example of sharing dependencies configured in the \texttt{webpack.config.js}, label=shared_mapping_wmf]
	  shared: share({
			"@angular/core": { 
				singleton: true, 
				strictVersion: false, 
				requiredVersion: '12.2.0' 
			},
			"@angular/common": { 
				singleton: true, 
				strictVersion: false, 
				requiredVersion: '12.2.0' 
			},
			"@fundamental-ngx/core": { 
				singleton: true, 
				strictVersion: false, 
				requiredVersion: '0.33.0-rc.214' 
			},
			
			...sharedMappings.getDescriptors()
		})
\end{lstlisting}

The listing \ref{shared_mapping_wmf} is an example of how to share libraries in a restrictive way. To provide a less restricted configuration a simple array of the shared dependency names suffices. But to ensure a redundant free landscape, these restrictions are necessary. Each configuration property will be explained below.

\begin{itemize}
	\item \texttt{singleton} - This property defines if the dependency should be able to be loaded more than once in different versions or not. If set to \texttt{true}, WMF will automatically pick the highest version of a major release of this dependency available and distribute it to the \texttt{remotes}.\cite{wmf_version_mismatch}
	
	\item \texttt{strictVersion} - This property defines if the dependency requires a specific version to work. If set to \texttt{true} WMF, will load the required version even if another dependency mapping with the same name is present. This can lead to conflicts with the \texttt{singleton} property, if configured poorly.
	
	\item \texttt{requiredVersion} - This property defines the required version of the dependency. When working with a package manager (e.g. NPM), this version has to be aligned with the locally installed version of the dependency. If the \texttt{strictVersion} property is set to \texttt{false}, this property defines the minimum version for the micro frontend. 
	
	It has to be mentioned that, WMF is able to distinguish between major releases. If a higher version of the same major release is available it will be loaded (e.g. \texttt{@angular/common@12.3.1}). For instance, if the next higher version is of a different major release e.g. 13.X.X, WMF would not consider to load it for the \texttt{remotes} which have the \texttt{requiredVersion} of release 12.X.X configured.
\end{itemize}

Now when it comes to sharing the dependencies of inside the micro frontend landscape, each \texttt{remote} has to bulk in. That means each micro frontend has to define its required dependencies in their respective versions. Additionally it has to be mentioned that the micro frontends themselves have to use dynamic imports when importing shared dependencies. Through the asynchronous behavior of the import, Webpack has time to pick the correct version of the dependency inside the landscape.\cite{wmf_concepts}
Analyzing this statement in combination with the information in the above list, it becomes obvious that multiple versions of the same framework can exist in a landscape. The below figure explains it in a graphical way.

\begin{figure}[!h]
	\centering
	\includegraphics[width=0.7\textwidth]{Figures/multi_version_diagramm.drawio.png}
	\caption{WMF way of handling multi-versions}
	\label{fig:wmf_multiversions}
\end{figure}

As it can be seen in \ref{fig:wmf_multiversions} WMF always picks the the highest major release, assuming the respective micro frontends have the above shown configuration \ref{shared_mapping_wmf} applied. If for instance, if \textbf{MF3} has its \texttt{strictVersion} property set to \texttt{true}, it would cause the loading of its libraries too.

When now looking at the introduced example, it is obvious that sharing several versions of the same framework over the whole landscape can not go without side effects. One of which is the increase in bundle sizes, since every \texttt{remote} bundles their local dependencies. WMF then picks the one to serve during the runtime of the landscape.
This impact has a trade-off tough. Returning users can benefit from cached dependencies but in certain use cases this is not acceptable.\cite{wmf_multi_versions}
 

	% \glsresetall
\chapter{Content Delivery Networks} % Main chapter title
\label{Chapter3}

\lhead{Chapter 3. \emph{Content Delivery Networks}}

One way of avoiding redundant libraries can be a central repository, where all resources are stored and can be requested via an API. Such a concept exists in the idea of Content Delivery Networks (CDNs).
Since the resources are loaded via the CDN API, a dependency is available under a unique URL. Loading a dependency twice means the browser requests the resource from the same URL with the same parameters. Therefore, the browser can determine if the given resource is already present and decide to either load it from the CDN or from the browser cache, if it is already loaded.\cite{caching_in_browser}

Nonetheless, a CDN has its limits which have to be considered. Costs can play a key role in the decision process.
Even though free CDN services like \textbf{Unpkg.com} or \textbf{cdnjs} are valid options, the advantages of using these might differ depending on the business case.
For instance, those services might solve the issue of redundant libraries in a micro frontend landscape, but do not necessarily improve the user experience of the application. 
Since the resources would be loaded from a cloud based service, the server location has significant influence on the latency of the requests. Therefore, depending from where the website is accessed, this circumstance could lead to a performance decrease when initially loading the page.\cite{cdn_general}

Also the technological dependency on a public cloud based CDN has to be considered. Most cloud based services are scalable and replicable. However, a down time could lead to immense costs for a platform provider relying on a public CDN to deliver the platform resources.
Besides, it is never guaranteed that certain dependencies are always available on the public CDN.

Another option is self-hosting a CDN and providing all required resources independently. A business would need to acquire hardware-resources (physically or via a hardware-as-a-service provider like Amazon Web Services), maintain the servers, fill the CDN with the necessary content and keep it up to date. However costs attached to this scenario can be immense and exclude this method due to cost inefficiency.\cite{Meassuring_a_commercial_CDN}

This chapter will give further information about the CDN technology, how it functions and what further features it offers to optimize a micro frontends or websites performance.

\section{Architecture}

The basic architecture of a CDN can be split into three different building blocks.

\begin{itemize}
	\item \textbf{Point of Presence (PoPs)} - Strategically located data centers around the world. Their function is to reduce the round trip time of requests. PoPs usually consist of several caching servers.
	\item \textbf{Caching servers} - These servers are located in different PoPs and serve the function of caching resources from the origin server. That way website loading times and bandwidth allocations are reduced.
	\item \textbf{Hardware, like SSD/HDD and RAM} - Located in the cache servers, the purpose of this building block is to provide the necessary storage and computing capacity. Better hardware means faster computing time, which then again improves the overall performance of the designated caching server.
\end{itemize}

Besides the above three building blocks, another one is crucial for the architecture of a CDN: The \textbf{origin server}. In a CDNs topology, the origin server can be compared to the center, or core. This is the server onto which the CDNs content is uploaded, synced with or distributed over the CDNs caching servers.\cite{cdn_origin_server}

An example for a basic CDN distribution is shown in figure \ref{fig:cdn_general_arch}.

\begin{figure}[!h]
	\centering
	\includegraphics[width=1\textwidth]{Figures/basic_cdn_arch.drawio.png}
	\caption{Basic distribution of a CDN}
	\label{fig:cdn_general_arch}
\end{figure}

It has to be mentioned that the geographical distribution in figure \ref{fig:cdn_general_arch} is simplified, for the sake of readability. In a real scenario, the cache servers of each PoP would themselves be distributed over different data centers on the continent. A client requesting resources from the CDN always communicates with the nearest possible cache server to their designated location.\cite{cdn_general}

\section{Design}

For a CDN to fulfill its purpose, four requirements have to be met. These are not only technical requirements, like necessary hardware, but rather concerning the general design of the CDN.
The four pillars of CDN design are:

\begin{itemize}
	\item \textbf{Performance} - First and foremost, the CDN has to provide a benefit when it comes to website performance. When the usage of a CDN increases the loading time of a website, the negative effect on user experience can cause financial harm to the host.\cite{cdn_general}
	\item \textbf{Scalability} - Since a cache server can serve multiple resources to different websites, it has to be able to handle traffic peaks. Without the aspect of scalability (either horizontal or vertical), this requirement cannot be met which would affect the CDNs performance.
	\item \textbf{Reliability} -  A website host has to rely on the CDN to deliver the websites resources. An outage on CDN side would cause the same effect to the relying websites, resulting in additional losses for the designated hosts. Therefore, most CDN providers commit to 99.9\% availability in their service level agreements (SLAs).
	\item \textbf{Responsiveness} -  The aspect of responsiveness targets the issue of synchronization inside the CDN. A CDN has to be capable to react to changes and distribute those to its PoPs over the globe accordingly. Otherwise inconsistencies could occur on the websites relying on the CDN. This is ensured through an automated pull mechanism, via which edge servers pull changes from the origin server.\cite{cdn_origin_server}
\end{itemize}

Additionally to those four requirements, the topology of a CDN has to be considered. There are two options which can be used. Both would serve their purpose, but have different advantages as well as disadvantages. 

\begin{itemize}
	\item \textbf{The scattered CDN} - This topology focuses on physical proximity. The PoPs are kept rather small in size, but are scattered around the world in greater frequency, thus providing as much proximity to their clients as possible. This topology excels in providing the CDN's resources into low-connectivity regions, since it is not highly dependent on wire infrastructure. When it comes to latency, it does not suffer as much due to the short distance between client and server. The trade-off with this topology is that it accrues more maintenance costs, since more PoPs are required to be maintained. Also, deploying new configurations can be connected to a lot of effort, depending on the number of PoPs scattered across the CDN. 
	Additionally, this number can also affect the RTT since every PoP inbetween the client and the server is a connection point.
	
	\item \textbf{The consolidated CDN} - Contrary to the scattered CDN, this topology is designed to consolidate its resources at strategically located data centers. Since the PoPs are only located in those major data centers, the servers available to the PoPs are highly advanced and provide a lot of hardware capacity. Additionally, since the number of major data centers around the world is rather limited, the number of PoPs is reduced as well, compared to the scattered topology. Following the quality over quantity principle, a PoP in this topology can handle a greater amount of traffic compared to its counterpart, and is also more resilient, specifically when it comes to DDoS attacks. Also, due to the moderate number of PoPs, it is easier and faster for the operators of the CDN to deploy new configurations.
	Nonetheless, the trade-off for this topology is that, even though it can handle a high number of requests, its reach to low-connectivity regions is rather limited. This is due to the proximity difference between the servers and clients. Additionally, deploying a new PoP into this topology requires more effort, since the PoPs are rather complex.	
\end{itemize}

The design decision is of course dependent on the business case for the CDN, as both topologies are intended to solve specific issues or challenges. \cite{cdn_architecture} 

\section{Optimization}

The CDN technology offers several ways of optimizing a websites performance and therefore, improving the user experience. In the following sections these optimizations will be showcased and explained.

\subsection{Route optimization with Anycast}

As previously mentioned, a CDN can be designed with different topologies which can affect its performance. Another factor which has to be considered in this equation is the routing itself. Basically, it does not matter if a cache server is located in close proximity to a client, if the requested resource is not present there. In this case, the client would have to be routed to different, cache servers located further away. 
To circumvent this, the Anycast routing is used in modern CDNs. This traffic routing algorithm is best explained in direct comparison with Unicast. Both serve the same purpose of routing requests to their designated destination, but they do it in different ways. Where with Unicast each node has a unique address, Anycast advertises multiple nodes with the same address.
For instance, in a Unicast orchestrated network the server address \texttt{10.10.0.1} would be only present once. Anycast, on the other hand, would advertise this exact address over multiple different servers around the globe. Thus, a request towards the address would reach its destination via the shortest path, given that the path will be identified and prioritized by devices that actually govern the flow of traffic.
The shortest path itself is counted in hops. Hops represent the number of time a request changes hands between hosts.\cite{cdn_route_opt}

\subsection{TLS Performance}

The route optimization with Anycast, also improves the RTT when using the TLS/SSL protocol. Since this section does not focus on explaining the protocol, only a short description is given.
SSL (Secure Socket Layer) or, as it now should be called, TLS (Transport Layer Security) is a protocol via which secure communications are ensured on the Internet. The communicating parties establish a connection via the following steps:

\begin{enumerate}[noitemsep]
	\item A so-called three-way handshake is done
	\item The parties agree upon an encryption method
	\item Mutual verification process is performed
	\item Symmetric keys for encoding and decoding are generated
\end{enumerate}

These steps are necessary to ensure secure communication and are a welcome trade-off for the benefits they provide. 

A CDN can provide improvement to this overhead and decrease the RTT of a request. Through the aspect of route optimization and general proximity, the overall request distance is decreased. Therefore, the RTT is shortened as well. The steps are still processed, they just do not have to travel that long. Additionally, the SSL/TLS negotiation process is shorter, too.\cite{cdn_ssl_tsl}

\subsection{Frontend optimization}

The term frontend optimization refers to the process of making a website more browser-friendly and reducing loading times. There are multiple ways to optimize a frontend. These will be explained under consideration of the role a CDN plays in them.

\begin{itemize}
	\item \textbf{Reducing HTTP requests} - When loading a website, the browser opens several HTTP connections, the number of which is actually limited by the browser. If a website requires more connections than a browser can open at the time, the browser has to start queuing the rest. This again leads to longer loading times and affects the user experience. A CDN improves on that by pre-pooling connections and ensuring they remain open throughout a session. Even though this does not reduce the actual number of requests, it does improve the response time for each one, making it so that every request can be processed faster. Additionally, HTTP/2 introduces the method of multiplexing. This allows a single TCP connection to transfer multiple different HTTP requests \cite{http2}.
	
	\item \textbf{File compression} - Not only do the number of requests or the proximity of the client to the server, affect the responsiveness of a website. The actual content has a major role in this aspect, too. Loading one single resource with a size of 1 GB takes a while, even if the server is close by. Reducing the size of this file or resource might increase the loading process. File compression like gzip is a method of doing exactly that. Most modern CDN providers offer automated file compression with gzip to reduce the actual content size delivered to the client.
	
	\item \textbf{Cache optimization} - Via caching, static files are stored either on the client device or in the cache of a nearby cache server. Locally stored static files do not have to be loaded via the network and are available to the browser for rendering almost immediately. The only question remaining is how long does a resource have to be cached. This information is necessary to optimize the use of the client's cache. The caching time is usually defined in the cache header of the request. Modern CDNs offer cache control options, which help in defining rules for exactly that header.
	CDNs have also started to use machine learning techniques to follow and understand content usage patterns and automatically optimize caching policies.
	 
	\item \textbf{Code minification} - Similar to the file compression method, the process of code minification offers a way of reducing file sizes too. Whereas a developer writes code in a humanly readable way, with spaces and line breaks, a machine does not need this kind of formatting. By removing comments, spaces and line breaks, the size of a code file can be reduced by 30\%. CDNs use methods like gzip, minify or a combination of these two to reduce the size of JavaScript, HTML or CSS files.
	
	\item \textbf{Image optimization} - Images can be immense in size and require a long time to load. The best way to display an image on a website would be to cache it first and then load it from the cache to reduce actual loading time through the network. Another option could be to reduce the actual size of the image and thus the loading time.
	However, other than code files, images are already compressed when loaded, therefore compressing them further to reduce file size might cause a loss of image quality. This is called lossy compression. If this trade-off is not an option, caching would be more effective.
	CDNs offer exactly that solution, caching images and providing them from the nearest source available to the client. If this does not suffice, CDNs also offer a progressive rendering option for images. On initially loading the page, the CDN would provide a lossy compressed version of the image quickly and then progressively replace it with higher-resolution variants.
	Alternatively, a website host could use vector or raster images. These are resolution independent, smaller in size and highly responsive.\cite{cdn_fe_opt_img_opt}
\end{itemize}

These methods in combination with the CDN technology provide a possible gain in user experience for a website host. \cite{cdn_fe_opt}

\section{Hosting a CDN}

Even though the method of a self-hosted CDN was not implemented in the developed prototype, it has to be considered for the context of this thesis. Since the basic principle of a CDN was already explained, this section will focus on the financial aspect of the technology.

This topic will be showcased based on a hypothetical scenario. The numbers and metrics of this scenario are either assumed or taken directly from the implemented prototype.

The first requirements of the scenario are the following:

\begin{itemize}[noitemsep]
	\item A micro frontend landscape has to be developed. 
	\item To reduce the runtime costs, the loading of redundant libraries should be avoided in the landscape.
	\item The libraries for the landscape were developed by the same team and are only available for internal use.
\end{itemize} 

These circumstances ensure that only a self-hosted CDN is applicable.
Next, the numbers and further requirements of the landscape are necessary. Those are required for the cost calculation of the CDN.
It is mentioned if the numbers are actually \textit{assumed/estimated} or \textit{taken from the prototype}.

\begin{itemize}[noitemsep]
	\item How often the site is accessed per day - $50 000$ times (estimated)
	\item Avg. byte size per page load without caching on client side - $8983.5$ KB (taken from the prototype)
	\item Avg. amount of GET requests to CDN per site load - $198$ requests (taken from the prototype)
	\item The site is only accessed in North America 
\end{itemize} 

Based on those requirements, the following values are calculated for the monthly CDN usage of the described scenario landscape.

\begin{quote}
	\begin{center}
		Let $A$ be the amount of requests to the CDN per month:
		\begin{math}
			A = 50000 \times 30 \times 198 = 297000000
		\end{math}
	\end{center} 
\end{quote}

\begin{quote}
	\begin{center}
		Let $B$ be the amount of bytes loaded per month:
		\begin{math}
			B = 50000 \times 30 \times 8983.5 = 13475250000 \: $KB$ \; = 13.47525 \: $TB$
		\end{math}
	\end{center} 
\end{quote}

Next, it will be calculated how much it would cost a company or department to pay for the CDN defined in this scenario.

There are multiple CDN solution providers on the market, including \textbf{Amazon CloudFront, Azure CDN, Google Cloud CDN}. \cite{top_10_cdn}
Since the pricing models of the mentioned providers differ, it is difficult to compare them directly. Applying the given scenario to the pricing calculator of the respective providers results in the following numbers.

\begin{itemize}[noitemsep]
	\item Amazon CloudFront - $1426.22$ \$ per month
	\item Google Cloud CDN - $1168.99$ \$ per month
	\item Azure CDN - $1123.90$ \$ per month
\end{itemize} 

After averaging those values an $Avg. price$ is calculated.  

\begin{quote}
	\begin{center}
		\begin{equation*}
			Avg. price = 
			\frac{
				\left(1426.22 + 1168.99 + 1123,90\right)
			}{
				3
			} = 1239.70 \: \$
		\end{equation*}
	\end{center} 
\end{quote}

It has to be mentioned that these values were calculated using the basic packages offered by the providers. There are further features which can be added to the solutions, which of course would increase the sum. Additionally, other providers have different pricing models, some of which include pricing depending on the HTTP requests sent to the CDN. Others charge prices when resources are stored in several PoPs around the world. Also, locations play a key role, as different regions pay more for traffic then others and this value is provider-specific.
Since it is not the goal to pick a provider here, but rather to provide a general overview of costs for such a solution, those additional features were not considered for the given scenario.




	% \glsresetall
\chapter{Web Components} % Main chapter title
\label{Chapter4}

\lhead{Chapter 4. \emph{Web Components}}

Another option to address the issue of redundancies are Web Components. Consisting of the following four standards, they provide reusable elements encapsulated in HTML tags.\cite{mdn_web_docs}

\begin{itemize}[noitemsep]
	\item Custom Elements
	\item Shadow DOM
	\item HTML Template
	\item ES Modules
\end{itemize}

The following sections will introduce the four standards of Web Components with their functions and showcase what purpose they can serve for the context of this thesis.

\section{Custom Elements}

This standard of Web Components provides an API via which new HTML tags can be defined and registered by the \texttt{CustomElementRegistry}. Since a single tag name can only be registered once, multiple registrations of the same element would lead to an exception, thereby making already registered tags reusable for the whole micro frontend landscape without reloading the code of the registered element.\cite{google_reusable_wcs}

\begin{figure}[!h]
	\centering
	\includegraphics[width=0.6\textwidth]{Figures/customElements_registered.drawio.png}
	\caption{Simple micro frontend landscape using Web Components}
	\label{fig:same_wc_example}
\end{figure}

Figure \ref{fig:same_wc_example} shows a simple micro frontend landscape with Luigi using Web Components. As displayed, both micro frontends are registering the same tag name \texttt{custom-element} using the \texttt{customElements} object. This object is a read-only property of the \texttt{Window} interface.
The scenario shown will cause a  \texttt{DOMException}, making it impossible to register the same tag twice.

The registration of Web Components follows the \textit{"first come, first served"} principle: The first micro frontend to register a tag name defines this tag and cannot be overwritten without reloading the page.\cite{mdn_web_docs_define}
Therefore, this standard is crucial for the goal to achieve in this transcript, by making impossible to register same components with the same tag name.

\section{Shadow DOM}

The DOM (Document Object Model) represents the elements of a markup document in a tree-like structure, consisting of connected nodes. The commonly used markup language for websites is HTML. \cite{wc_shadow_dom}
The Shadow DOM is also a DOM, but is attached to the actual DOM of the document. Underneath it, elements can be defined the same way as they are in the regular DOM. The difference appears during the rendering of the document, when a page is loaded. The Shadow DOM elements are rendered separately from the DOM it is attached to.\cite{simon_thesis}

To understand the relationship between the two connected DOMs following terms have to be explained.

\begin{itemize}
	\item \textbf{Shadow host} - The attachment of the Shadow DOM to the normal DOM happens via a node inside the normal DOM.
	\item \textbf{Shadow tree} - Since the Shadow DOM is a DOM in itself, it consists of nodes in a tree-like structure.
	\item \textbf{Shadow boundary} - The Shadow DOM capsules its Shadow tree an renders it separately from the actual DOM. This encapsulated area defines where the Shadow DOM begins and ends.
	\item \textbf{Shadow root} - Just like a regular DOM a Shadow DOM has a root from where it originates.
\end{itemize} 

Figure \ref{fig:shadow_dom} visualizes the relations between the newly introduced terms.

\begin{figure}[!h]
	\centering
	\includegraphics[width=1\textwidth]{Figures/shadow_dom.drawio.png}
	\caption{Shadow DOM architecture}
	\label{fig:shadow_dom}
\end{figure}

Through the isolation of the Shadow DOMs code, this standard offers a way to provide scoped HTML and CSS code to custom elements. As mentioned before, the nodes of the Shadow DOM are rendered separately. Therefore, styles, ids, names or even CSS classes and other configurations applied to a tag inside the Shadow DOM are not applied to the actual DOM.

%Listing \ref{list:shadow_dom_definition} provides be a simple example.
%
%\begin{lstlisting}[caption=Definition of a custom element using the Shadow DOM \cite{simon_thesis}, label=list:shadow_dom_definition,  xleftmargin=.0\textwidth, xrightmargin=.0\textwidth]
%<!DOCTYPE html> 
%	...
%	<body>
%		<p>Plain</p> 
%		<custom-element></custom-element>
%		
%		<script>
%			class CustomElement extends HTMLElement {
%				constructor() { 
%					super();
%					const shadow = this.attachShadow({mode: 'open'}); 
%					shadow.innerHTML = `<style> p {color: red;} </style>`; 
%					const shadowParagraph = document.createElement('p'); 
%					shadowParagraph.textContent = 'Blue'; 
%					shadow.appendChild(shadowParagraph);
%				}
%			}
%			customElements.define('custom-element', CustomElement); 
%			customElement = new CustomElement(); 
%			console.log(customElement.shadowRoot)
%		</script>
%	</body>
%</html>
%\end{lstlisting}
%
%The listing shows the usage of the Shadow DOM in combination with registering a custom element (named \texttt{custom-element}) using the customElements API.
%The \texttt{script} tag of the snippet starts with creating a new HTML element called CustomElement. It is an extension to the HTMLElement class. This new element has the Shadow DOM attached to it. Therefore, it is the Shadow host in the DOM tree. Via the \texttt{innerHTML} attribute a Shadow DOM Node is created. In this case it is a styling for paragraphs,followed by the creation of a paragraph onto which the styling is applied. Additionally the text color of the paragraph is set to blue and lastly it is attached to the Shadow root.
%The last three lines of the script, are the definition and registration of the new \texttt{custom-element}, the creation of a new \texttt{customElement} object and the output of the attached Shadow DOM in the console.
%
%Now when looking at the body of the above snippet, it becomes visible that beside the \texttt{custom-element} another paragraph is defined in the document. Usually the styles defined in the \texttt{script} below would apply to this paragraph, too. Through the encapsulation of the two separate DOMs this is not the case. The paragraph defined in the body of the document will be completely unaffected by the changes applied to the paragraphs in the Shadow DOM.\cite{simon_thesis}
%
%Now another configuration which is shown in the snippet is the \texttt{this.attachShadow({mode: 'open'})}. This mode property defines the visibility of the Shadow DOM to the browser. If set to \texttt{open}, the elements of the Shadow DOM can be inspected inside the sources. Thus, the \texttt{console.log(customElement.shadowRoot)} line would work if executed. 
%If the mode would be set to \texttt{closed}  the \texttt{console.log()} command in line 20 would return \texttt{null}.\cite{simon_thesis}

The Shadow DOM, even though separated from the DOM, has an attribute via which it can be made accessible. By adding the \texttt{mode} attribute to the \texttt{this.attachShadow()} method, the visibility of the Shadow DOM, to the document can be defined \cite{simon_thesis}. Possible values for the \texttt{mode} property are:

\begin{itemize}
	\item \textbf{open}
	\item \textbf{closed}
\end{itemize}

This behavior is not meant to be used as a measure of security, since it can be overwritten. The Shadow DOM encapsulates every part of its DOM elements, that means HTML, CSS and JavaScript. The \texttt{document} object, available during runtime, stays the same for the regular DOM as for the Shadow DOM though. Therefore, each configuration done in the Shadow DOM via scripts can be easily overwritten from any other script in the document, thus the mode can be changed even if initially set to \texttt{closed}.\cite{shadow_dom_encapsulation} For Web Components in particular it is not recommended to use this mode at all, since it would make them less flexible for end users.\cite{wc_shadow_dom_google}

For the context of this thesis, the Shadow DOM standard offers means to individualize the registered Web Components of a landscape. It can be used to apply customizations or add individual event listeners to specific components. This comes with a trade-off though. The level of isolation increases through the encapsulation of code elements between the single components. This increase makes it difficult to guaranteed that e.g. the same CSS class is not rendered in multiple different Shadow DOMs. Even though the redundancies would occur on code level and not on dependency level, they are redundancies nonetheless.

\section{HTML Template}

This standard offers a way to define reusable markup code. As a part of the HTML standard itself, the \texttt{template} tag is used to define templates. These are not rendered unless used by another element. Similar to \texttt{template}, \texttt{slot} serves the same purpose, but in a different way. Templates are defined HTML code snippets which can be cloned and inserted in other document elements or even elements rendered in the Shadow DOM.
Slots on the other hand serve as placeholder for either default markup texts or other DOM elements. Therefore, a template is a rather static piece of reusable HTML code, compared to slots.
Slot themselves are identified by their name, the content is inserted when the slot is addressed by its name.
Listings \ref{list:template_example} and \ref{slot_example} provide examples how exactly these two standards are used. \cite{wc_html_template_slots}

\begin{lstlisting}[caption=Definition and usage of the \texttt{template} standard \cite{wc_html_template_slots}, label=list:template_example,  xleftmargin=.0\textwidth, xrightmargin=.0\textwidth]
<!-- Definition of the template -->
<template id="my-paragraph">
	<p>My paragraph</p>
</template>

<!-- Usage of the template in an Web Component -->
customElements.define('my-paragraph',
class extends HTMLElement {
	constructor() {
		super();
		let template = document.getElementById('my-paragraph');
		let templateContent = template.content;
		
		const shadowRoot = this.attachShadow({mode: 'open'})
		.appendChild(templateContent.cloneNode(true));
	}
});
\end{lstlisting}

It is important to note, that the defined template in the listing is not rendered unless somehow included in a DOM (either Shadow DOM or the regular DOM), via JavaScript.

\begin{lstlisting}[language=HTML5, caption=Definition and usage of the \texttt{slot} standard \cite{wc_html_template_slots}, label=list:slot_example,  xleftmargin=.0\textwidth, xrightmargin=.0\textwidth]
<!-- Definition of the slot -->
<p>
	<slot name="my-text">Default input</slot>
</p>

<!-- Usage of the slot in the markup document -->
<my-paragraph>
	<ul slot="my-text">
		<li>Some different input</li>
		<li>In a list!</li>
	</ul>
</my-paragraph>
\end{lstlisting}

As it can be seen in listing \ref{list:slot_example}, the \texttt{slot} definition has some default content defined. In the above case it´s a simple text. When the slot is used, this content is overwritten by the content in the element which is calling the \texttt{slot} by its name.
In that case the content is replaced by some list items.
Other than the templates, slots are always rendered if included in the markup via their respective names. The content which is rendered is depended on, if the default content is overwritten or not.

Even though slots are a HTML standard just as templates, their support for browsers is not always guaranteed. This is due to the fact, that compared to templates it is a rather new standard.

In combination these two standards offer a way to define flexible, reusable markup code for Web Components.\cite{wc_html_template_slots} 

When developing own Web Components for a micro frontend landscape, this aspect can be used to reduce repetitive HTML code elements by serving them a templates. Similar to the Shadow DOM this feature could be used to apply more flexibility to the Web Components, by adding placeholders for customization. But contrary to the Shadow DOM, this does not increase the level of isolation, since the rendered elements are not separated from the actual DOM. Redundant code snippets can be defined as reusable templates, thereby reducing the redundancies in the code itself.

\section{ES Modules}

The last standard is not referred to by every source, but according to \cite{wc_specifications}, it is a stable part of the Web Components standard, via which Java Script modules can be defined and reused by other documents.
Thus the development of Web Components can be done in a modular way, making every component available to other documents, using the \texttt{type="module"} attribute.

\begin{lstlisting}[language=HTML5, caption=Importing modular Java Script documents into another \cite{wc_specifications}, label=list:es_modules_example,  xleftmargin=.0\textwidth, xrightmargin=.0\textwidth]
<!-- Import of the JS Module -->
<script type="module" src="awesome-explosion.js"></script>
	...
<script type="module">
	import 'awesome-explosion.js';
	...
	import {awesomeExplosion} from '@awesome-things/awesome-explosion';
</script>

<!-- Usage of the newly imported module -->
<awesome-explosion>
	...
</awesome-explosion>
\end{lstlisting}

Listing \ref{list:es_modules_example} shows such an import. Assuming the \texttt{awesome-explosion.js} files contains the definition of an element called \texttt{awesome-explosion}, these lines enable the document to use this element.\cite{wc_specifications}

This feature is essential for the prototype developed, since via this aspect the components are made available to the landscape. The used Web Components are served as ES Modules to the landscape and are imported similarly as shown in \ref{es_modules_example}. Therefore, this aspect enables a component to be imported in multiple micro frontends as a module.

\section{Additional considerations}

This chapter has shown, that this technology solves the issue of redundancies in a micro frontend landscape, on a different level. Instead of reducing the redundant libraries, it actually reduces the code itself used by the landscape.

There is still a niche case which can be raised as an objection though. Since the registration and therefore the naming of elements is up to the developers, it can occur that two components are registered under the same tag name, but the elements themselves are different. In this case, it is up to the developers to use the standard properly.

Assuming a common micro frontend landscape where every micro frontend is developed by independent, isolated teams, using heterogenic frameworks, the mentioned issue might occur. Two teams use Web Components and try to register different elements under the same tag name in the same landscape. This requires organizational interference on team level.

One way would be to define a namespace for every team when creating Web Components e.g. Team 1 has the namespace \texttt{team-1}, causing a Web Component registered by the micro frontend of Team 1 to be named \texttt{team-1-tagname}. This could lead to redundancies, because there is no way of assuring that the \texttt{team-1-tagname} and \texttt{team-2-tagname} Web Components are not the same.\cite{wc_best_practices}

A better way might be to assign a common Web Components library like \textbf{UI5 Web Components}. That way, the registered elements are provided by the library and are limited to a set of unique registrable tags. If another micro frontend would try and register an already registered element of the library, a warning is thrown but no error occurs.

Most Web Component libraries also offer so called scoping options. This feature is employed, when versions of the used components differ between micro frontends. It enables the developers to customize their Web Components and register them under different tags. With this feature, it is made possible to register components according to their respective versions, by adding a version-specific suffix. That, of course, might lead to redundancies again. But it also reduces the dependency of the developer teams to always use the latest version of the component library or the first registered element in the landscape. \cite{ui5_webcomponents_scoping} \cite{openwc_scoping}




	% \glsresetall
\chapter{Webpack Module Federation} % Main chapter title
\label{Chapter5}

\lhead{Chapter 5. \emph{Webpack Module Federation}}

A rather new way of creating micro frontends is the \textbf{Webpack 5 Module Federation (WMF)}. With the release of version 5 on October 10, 2020, this technology added certain features which improved its usage for developing micro frontends.\cite{wmf_concepts}
The way it does that, is via modularizing self-compiled code parts and publishing them for integration by other modules. This published modules can be micro frontends themselves and are called \textbf{remotes} whereas the integrating modules are called \textbf{hosts}. 
\textbf{Hosts} refer to \textbf{remotes} under a configured name. This name is not actually known to the \textbf{host} during the compile time, but is first resolved at runtime.
The self-compiled \textbf{remote} in this case can be anything, a micro frontend or some sort of utility script. This way the Module Federation provides a way to avoid external or manual script loading and instead gives opportunities to automatically lazy load necessary code blocks during runtime.\cite{wmf_concepts}
The usage of the WMF is tied to the Webpack bundler, since the necessary configuration is done in the \texttt{webpack.config.js}. This restriction is applied to every \textbf{remote} in the WMF landscape, not only to the \texttt{host}.

Since the main focus of this document is to propose methods of avoiding redundancies in micro frontend landscapes, an introduction of this WMF feature is given here.
WMF can be used in combination with most of the common UI Frameworks. Since the implementation for this thesis was done with Angular, the further examples and explanations will be Angular-based.

\section{Enabling the Module Federation}

Prior to introducing the usage of the Module Federation, it is necessary to introduce Webpack itself first, as it is a mandatory feature for using the Module Federation. Popular UI frameworks like React, VueJS or Angular use Webpack under the hood anyway, so it isn't as much of a restriction as it seems.\cite{webpack_angular}\cite{webpack_react}\cite{webpack_vue}
The documentation of the named frameworks imply that Webpack is used by default, but can be customized if necessary by the developer.

For Angular in particular, it is necessary to install two dependencies via the Angular CLI, to enable the features of the Module Federation. The command used for this is shown in \ref{list:angluar_wmf_command}. 

\begin{lstlisting}[language=Bash, caption=Angular CLI console command to enable Module Federation in an Angular project, label=list:angluar_wmf_command,  xleftmargin=.0\textwidth, xrightmargin=.0\textwidth]
	ng add @angular-architects/module-federation --project name --port port
\end{lstlisting}

These commands enable the Module Federation for an Angular project. Since the CLI protects the Webpack configuration from access, a custom builder is required. The \texttt{@angular-architects/module-federation} package provides exactly that.
After installing this dependency in an Angular project, a \texttt{webpack.config.js} will appear on root level of the corresponding project.\cite{wmf_angular_dependency_install}
This dependency has to be added in each \textbf{remote} or \textbf{host} of the WMF landscape. 

After enabling the Module Federation inside a project, the necessary configuration can be applied to the \texttt{webpack.config.js} file. The \textbf{remotes} publish their modules and \textbf{hosts} consume them. Thus a developer can distinguish what module has which role.

\section{Shared dependency feature}

As noted before WMF is a micro frontend framework, which offers means to solve the issue of redundant libraries in its landscapes.
This feature is enabled and configured, as well as the rest of the WMF, via the \texttt{webpack.config.js}. When configuring the components of the landscape, it is possible to define a section where shared dependencies are described. These dependencies can be defined in different ways. For instance, it is possible to define a strict version of the dependency, which would result in the framework loading this specific version. Or one can define a less restricted dependency, which would mean that if another \texttt{remote} loads the same dependency but in a different version, the framework would automatically apply the highest major version of the dependency to both micro frontends.

\begin{lstlisting}[language=JavaScript, caption=Example of sharing dependencies configured in the \texttt{webpack.config.js}, label=list:shared_mapping_wmf,  xleftmargin=.01\textwidth, xrightmargin=.01\textwidth]
	shared: share({
		"@angular/core": { 
			singleton: true, 
			strictVersion: false, 
			requiredVersion: '12.2.0' 
		},
		"@angular/common": { 
			singleton: true, 
			strictVersion: false, 
			requiredVersion: '12.2.0' 
		},
		"@fundamental-ngx/core": { 
			singleton: true, 
			strictVersion: false, 
			requiredVersion: '0.33.0-rc.214' 
		},
		
		...sharedMappings.getDescriptors()
	})
\end{lstlisting}

Listing \ref{list:shared_mapping_wmf} is an example of how to share libraries in a restrictive way. To provide a less restricted configuration, a simple array of the shared dependency names suffices. But to ensure a redundant free landscape, these restrictions are necessary. Each configuration property will be explained below.

\begin{itemize}
	\item \texttt{singleton} - This property defines if the dependency should be able to be loaded more than once in different versions or not. If set to \texttt{true}, WMF will automatically pick the highest version of a major release of this dependency available and distribute it to the \textbf{remotes}.\cite{wmf_version_mismatch}
	
	\item \texttt{strictVersion} - This property defines if the dependency requires a specific version to work. If set to \texttt{true} WMF, will load the required version even if another dependency mapping with the same name is present. This can lead to conflicts with the \texttt{singleton} property, if configured poorly.
	
	\item \texttt{requiredVersion} - This property defines the required version of the dependency. When working with a package manager (e.g. NPM), this version has to be aligned with the locally installed version of the dependency. If the \texttt{strictVersion} property is set to \texttt{false}, this property defines the minimum version for the micro frontend. 
	
	It has to be mentioned that WMF is able to distinguish between major releases. If a higher version of the same major release is available, it will be loaded (e.g. \texttt{@angular/common@12.3.1}). For instance, if the next higher version is of a different major release e.g. 13.X.X, WMF would not consider to load it for the \textbf{remotes} which have the \texttt{requiredVersion} of release 12.X.X configured.
\end{itemize}

Now, when it comes to sharing the dependencies inside the micro frontend landscape, each \textbf{remote} has to participate. That means each micro frontend has to define its required dependencies in their respective versions. Additionally, it has to be mentioned that the micro frontends themselves have to use dynamic imports when importing shared dependencies. Through the asynchronous behavior of the import, Webpack has time to pick the correct version of the dependency inside the landscape.\cite{wmf_concepts}
Analyzing this statement in combination with the information taken from \ref{list:shared_mapping_wmf}, it becomes obvious that multiple versions of the same framework can exist in a landscape. 

\section{Multi-version landscapes in WMF}

Figure \ref{fig:wmf_multiversions} illustrates the case, of how WMF handles a multi-version landscape.

\begin{figure}[!h]
	\centering
	\includegraphics[width=0.7\textwidth]{Figures/multi_version_diagramm.drawio.png}
	\caption{WMF way of handling multi-versions}
	\label{fig:wmf_multiversions}
\end{figure}

As it can be seen, WMF always picks the latest major release, assuming the respective micro frontends have a similar configuration as shown in listing \ref{list:shared_mapping_wmf} applied. If, for instance, \textbf{MF3} has its \texttt{strictVersion} property set to \texttt{true}, it would cause the loading of its libraries too.

There are side effects to sharing the same dependency over the whole micro frontend landscape. One of which is the increase in bundle sizes, since every \textbf{remote} bundles its local dependencies. WMF then picks the one to serve during the runtime of the landscape.
This impact has a trade-off tough. Returning users can benefit from cached dependencies.\cite{wmf_multi_versions}

Integrating several \textbf{remotes} using the same dependencies leads to the issue of redundancies, which WMF is able to resolve. Via configuration of shared dependencies WMF provides a way to reduce redundant libraries in its landscape.
	% \glsresetall
\chapter{Development of the prototypes} % Main chapter title
\label{Chapter6}

\lhead{Chapter 6. \emph{Development of the prototypes}}

This chapter will provide an overview of the prototype landscapes and explain how the previously introduced technologies were used in them.

\section{Definition of representation}

Before the implementation is showcased, it is explained why and how representation is defined for the context of the thesis.
Since not every micro frontend landscape is the same, a general definition was required before developing the prototypes.
For clarification, the term prototype is a synonym for landscape and vise versa, in the following context.
Those prototypes had to fulfill most of the following requirements, to be considered representative:

\begin{enumerate}
	\item The landscape has to contain at least 6 or more micro frontends
	\item The different micro frontends in the landscape, have to load the same libraries/dependencies to cause redundancies
	\item The tech stack of the landscape has to be heterogenic (not only one UI framework is used).
	\item The embedded micro frontends have to be isolated projects
	\item The landscape contains different versions of a dependency
\end{enumerate}

In specific cases, not all requirements could be met, it is explained why though.

\section{Prototype overviews}

This section will shortly describe the developed landscapes.

\subsection{Prototype 1 - CDN/NPM}

The first prototype developed was the CDN/NPM landscape. Totaling 8 micro frontends embedded as iFrames into a Luigi-Core-React application.
Figure \ref{fig:unpkg_prototype_architecture} visualizes this environment.
 
\begin{figure}[!h]
	\centering
	\includegraphics[width=0.7\textwidth]{Figures/unpkg.architecture.drawio.png}
	\caption{Overview of the Luigi prototype using Unpkg.com}
	\label{fig:unpkg_prototype_architecture}
\end{figure}

This prototype represents a Luigi micro frontend landscape, where each Luigi node is a single-page application (SPA). 
The 4 NPM apps, embedded in the landscape, were implemented for comparison. They are meant to show, how the the CDN technology performs, in direct comparison to a bundled application.
In their core, each application contains the same UI-elements. Thus, it is made possible to compare them, in terms of loading times and bytes loaded.
The goals to show with this environment were:

\begin{itemize}
	\item How redundancies occur in a micro frontend landscape
	\item That the browser can´t distinguish between bundled resources, contrary to resources loaded from the CDN
	\item If the UI framework, has an effect on the performance of the site
\end{itemize}

This landscape, fulfills 4 out of the 5 requirements, missing the aspect of different versions. The reason behind this is that, even though it is possible to request different versions of a resource from the CDN, the CDN itself does not offer any means to resolve multiple versions inside the landscape. That means that the library is imported 2 times in different versions. That just increases the redundancies inside the landscape. Due to this predictability, this aspect was not considered for that landscape.

\subsection{Prototype 2 - WMF/Web Components}

The second prototype developed was the WMF and Web Components landscape. It contains 36 micro frontends in total, which are split over 6 Luigi Nodes.
Each Luigi Node, is therefore a compound of micro frontends.
Figure \ref{fig:compound_prototype_architecture} provides an overview of the Node landscape.

\begin{figure}[!h]
	\centering
	\includegraphics[width=0.7\textwidth]{Figures/compound_views_overview.drawio.png}
	\caption{Overview of the Luigi Nodes in the WMF/Web Components environment}
	\label{fig:compound_prototype_architecture}
\end{figure}

The 6 Nodes displayed in figure \ref{fig:compound_prototype_architecture} contain 6 micro frontends each. 
Figure \ref{fig:compound_wmf_single_node} provides an overview on how the micro frontends are arranged inside a Node to a compound. 

\begin{figure}[!h]
	\centering
	\includegraphics[width=0.7\textwidth]{Figures/compound_wmf_single_node.drawio.png}
	\caption{Example overview of a single Node inside the WMF/Web Components environment}
	\label{fig:compound_wmf_single_node}
\end{figure}

Even though these two technologies are embedded in the same Luigi dashboard, they are referred to as separate landscapes in certain cases. This is due to used technologies, the first one be the Web Components.

\subsubsection{Embedding a micro frontend into a compound in Luigi}

The micro frontends in the 3 Web Component Nodes are embedded as part of compounds. The according configuration is made in the \texttt{luigi.config.js}.\cite{luigi_wc} \cite{luigi_compound}
Contrary to WMF, this configuration allows to embed the referred components in so called \texttt{compoundItemContainer}. The arrangement and number of columns inside the Node are defined in the \texttt{luigi.config.js}.
Listing \ref{list:compound_luigi_config} shows the exact configuration for that.

\begin{lstlisting}[caption=Configuration of Web Components Node for a compound view, label=list:compound_luigi_config,  xleftmargin=.0\textwidth, xrightmargin=.0\textwidth]
Luigi.setConfig({
	navigation: {
		nodes: [{
			label: 'Home',
			pathSegment: 'home',
			icon: 'home',
			hideFromNav: true,
			globalNav: true,
			hideSideNav: true,
			defaultChildNode: 'mixedVersions',
			children: [
				// Mixed Versions
				{
					label: 'Luigi Compound mixed versions',
					pathSegment: 'mixedVersions',
					icon: 'activate',
					compound: {
						renderer: {
							use: {
								extends: 'grid',
								createCompoundItemContainer: (itemConfig, 
																						containerConfig, 
																						superRenderer) => 
								{
									const itemContainer = superRenderer
									.createCompoundItemContainer(itemConfig);
									if (!itemConfig.noPadding) {
										itemContainer.style = itemContainer
										.getAttribute('style') + 
										' ; padding: 20px ; overflow: hidden';
									}
									return itemContainer;
								}
							},
							config: {
								columns: '1fr 1fr 1fr',
								layouts: [{
									minWidth: 0,
									maxWidth: 1024,
									columns: '3fr',
									gap: 0
								}]
							}
						},
						children: [
						// MFEs for this compound go here
						{
							webcomponent: true,
							viewUrl: 'URL to the view to be embedded',
							label: 'A label for the MFE',
							layoutConfig: {
								column: "auto"
							}
						},
						...
						]
					}
				},
\end{lstlisting}

Line 14 to 16 define the central Node navigation element for the Luigi dashboard. In line 17 it is defined that the content of this Node is to be a compound. The following line 18 defines the renderer for the content and the tools are the properties of the \texttt{use} property in line 19. Line 20 to 34 define the actual CSS attributes of the to be rendered container for the compound elements. The arrangement of the container , is managed in the \texttt{config} property in line 35. In this scenario, the containers can be arranged in 3 columns, each one taking 1 fraction of 1024 pixels.
At line 45 the actual compound elements or micro frontends are defined as an array of JavaScript objects.

\subsubsection{Embedding a micro frontend into a compound in WMF}

Contrary to the Web Component Nodes, the WMF Nodes are actually embedded as iFrames into the Luigi dashboard. Therefore, the Luigi configuration for that landscape, is similar to the CDN/NPM environements. The listing \ref{list:wmf_luigi_config}, shows an example on how it was done.

\begin{lstlisting}[caption=Example configuration to embed a WMF micro frontend compound as a Node in Luigi, label=list:wmf_luigi_config,  xleftmargin=.0\textwidth, xrightmargin=.0\textwidth]
	Luigi.setConfig({
		navigation: {
			nodes: [{
				label: 'Home',
				pathSegment: 'home',
				icon: 'home',
				hideFromNav: true,
				globalNav: true,
				hideSideNav: true,
				defaultChildNode: 'sameVersions',
				children: [
                {
					pathSegment: 'wmfHostSame',
					label: 'WMF Compound Same',
					icon: 'technical-object',
					viewUrl: 'https://angular-wmf-same-shell.surge.sh/',
					loadingIndicator: {
						enabled: false
					}
				},
			...
\end{lstlisting}

As it can be seen, the integration is done via an URL. The same view displayed in the dashboard, can be found under the referred URL. This means, to create a compound view in WMF, the creation of the necessary container, the arrangement and the actual placements of the micro frontends, is managed by the host application of the WMF environment and not Luigi.
The code for these development, is shown and explained in section \ref{wmf_implementation_prototype} of this chapter.

\section{CDN - Unpkg.com} 

Chapter \ref{Chapter3}, introduced the concept of the CDN. A remote server is provides the necessary platform resources via an API, thus avoiding the necessity of bundling those resources.
To evaluate the impact of this technology on a micro frontend landscape, the first prototype was developed using the public cloud based CDN Unpkg.com. It is an open-source project, built and maintained by Michael Jackson. It runs on the Cloudflare platform, and auto-scalable servers are provided by Fly.io, which are located in 17 cities around the world.\cite{unpkg_doc}

An open API is available, through which resources can be requested. Via path and query parameters in the URL, necessary information like the dependency version can be provided.

The usage of the Unpkg.com-CDN is embedded in the code. Instead of loading the dependencies via the local \texttt{node\_modules} directory, they are directly loaded using the CDNs API, the usage of which is displayed in the listing \ref{list:unpkg_import}.

\begin{lstlisting}[caption=Import of a dependecy using the unpkg API, label=list:unpkg_import]
	<script src="https://unpkg.com/@ui5/webcomponents@1.0.0-rc.15/dist/StandardListItem.js?module" type="module"></script>
\end{lstlisting}

This script tag is placed in the central \texttt{index.html}. The loaded resource can be read from the URL. The first part of the URL refers to the protocol and the host. The first path parameter is the npm dependency name. In the case of \ref{unpkg_import} it is \texttt{@ui5}, followed by a subdirectory with its respective version annotated with an @.
As it is described in the documentation, the CDN supports two query parameters.

\begin{itemize}[noitemsep]
	\item \texttt{?meta} - to request meta data about the loaded package, in a JSON format
	\item \texttt{?module} - to expand all import specifiers in JavaScript modules to unpkg URLs
\end{itemize}

If the import is handled via a script tag, as it is done in \ref{list:unpkg_import}, the \texttt{type="module"} attribute has to be added, depending on the resource loaded.
In the case of the given example, the resource is a JavaScript module type. To ensure it is parsed by the runtime as such, this attribute has to be added.\cite{js_module_type}

\section{Web Components - UI5 Web Components} 

The Web Components standard was partially used in the development of the second prototype. For development itself no UI application framework was used, which means all three compounds were developed using plain JavaScript. The actual components used in the micro frontends of the compounds, were taken from a component library called UI5 Web Components. Each element in that library is in fact a Web Component.\cite{ui5_wc_github}
Consisting of the picked elements and under the usage of the scoping feature of the component library, the compounds where created. This made it possible to scope the tags of the UI5 Web Components.\cite{ui5_webcomponents_scoping}

\begin{lstlisting}[caption=Scoping feature used in the prototype, label=list:scoping_wc_prototype,  xleftmargin=.0\textwidth, xrightmargin=.0\textwidth]
import { LuigiElement, html } 
from "@luigi-project/client/luigi-element.js";
import { setCustomElementsScopingSuffix } 
from "@ui5/webcomponents-base/dist/CustomElementsScope.js";
setCustomElementsScopingSuffix("placeholder");
import "@ui5/webcomponents/dist/Dialog.js";
import "@ui5/webcomponents/dist/Button.js";
[...]

export default class extends LuigiElement {
	constructor() {
		super();
		[...]
		render() {
			return html`
			<div>
			<div class="header">
			<h2>Products table - Version: placeholder</h2>
			</div>
			
			<ui5-table-placeholder id="ui5-table" 
			ui5-table sticky-column-header>
			<!-- Columns -->
			<ui5-table-column-placeholder 
			slot="columns" 
			style="width: 4rem">
			<span 
			style="line-height: 1.4rem">
			Product
			</span>
			</ui5-table-column-placeholder>
			
			[...]
			
			</ui5-table-placeholder>
			</div>`;
		}
	}
\end{lstlisting}
	
Listing \ref{list:scoping_wc_prototype} shows the exact usage of the scoping feature. 
The imported \textit{setCustomElementsScopingSuffix} function enables to define a custom suffix to all UI5 Web Component elements, if not configured otherwise. Also, as it can be seen in the same listing, the suffix is set to \textit{placeholder} using the imported method \textit{setCustomElementsScopingSuffix("placeholder")}.
This is done for the sake of the experiment itself. It was intended to deploy six similar looking micro frontends, using Web Components, into the Luigi landscape, to test how many redundancies occur. In addition it was meant to be tested how different versions of the same elements could be registered. For example the \textit{Bar} element in version \textit{1.0.1} is registered under same tag as the same element in version \textit{1.1.0}. This scenario has to be handled and the scoping feature is one way of doing so.
By defining a version suffix for the \textit{Bar} element tag, the browser can distinguish those elements.
In case of the prototype to create this exact scenario, one global tag suffix \textit{placeholder} was picked and later replaced. The replacement itself happened during the bundling of the project. The RollUp bundler provided the necessary functionality for that.
Inside the \texttt{rollup.config.js} several configurations were defined, which are shown in listing \ref{list:rollupconfigjs}.
	
	\begin{lstlisting}[language=JavaScript, caption=Content of the \texttt{rollup.config.js}, label=list:rollupconfigjs,  xleftmargin=.01\textwidth, xrightmargin=.01\textwidth]
		imp\part{ort resolve from '@rollup/plugin-node-resolve';
			import json from '@rollup/plugin-json';
			import url from "@rollup/plugin-url";
			import { terser } from "rollup-plugin-terser";
			import replace from '@rollup/plugin-replace';
			import { SameVersions } from './rollup_files/same_version';
			import { DiffVersions } from './rollup_files/different_version';
			import { MixedVersions } from './rollup_files/mixed_version';
			
			let buildArray = [];
			
			function aggregateConfigs() {
				for(let buildConfig of SameVersions) {
					buildArray.push(buildConfig);
				}
				
				for(let buildConfig of DiffVersions) {
					buildArray.push(buildConfig);
				}
				
				for(let buildConfig of MixedVersions) {
					buildArray.push(buildConfig);
				}
			}
			
			aggregateConfigs();
			
			export default buildArray;
		\end{lstlisting}
		
		The configuration of the bundler was split into several Java Script files, to improve the readability of the configuration file itself. The content of such a file can be seen in \ref{list:rollupconfigfile}.
		
		\begin{lstlisting}[language=JavaScript, caption=Actual configuration for the \texttt{rollup.config.js}, label=list:rollupconfigfile, xleftmargin=.01\textwidth, xrightmargin=.01\textwidth]
			import resolve from '@rollup/plugin-node-resolve';
			import json from '@rollup/plugin-json';
			import url from "@rollup/plugin-url";
			import { terser } from "rollup-plugin-terser";
			import replace from '@rollup/plugin-replace';
			
			export const MixedVersions = [
			// 1
			{
				input: 'src/tableView.js',
				output: {
					file: 'dist/tableViewMixedVersions1.js',
					format: 'es',
					compact: true
				},
				plugins: [
				replace({
					'placeholder': '0-9-0',
				}),
				terser(),
				resolve(),
				json(),
				url({
					limit: 0,
					include: [
					/.*assets\/.*\.json/,
					],
					emitFiles: true,
					fileName: "[name].[hash][extname]",
					publicPath: "\" + new URL(\".\", import.meta.url) + \"", // relative configuration for assets (TBD with UI5 Web Components team)
				})
				]
			},
			// 2
			{
				input: 'src/tableView.js',
				output: {
					file: 'dist/tableViewMixedVersions2.js',
					format: 'es',
					compact: true
				},
				plugins: [
				replace({
					'placeholder': '1-1-0',
				}),
				terser(),
				resolve(),
				json(),
				url({
					limit: 0,
					include: [
					/.*assets\/.*\.json/,
					],
					emitFiles: true,
					fileName: "[name].[hash][extname]",
					publicPath: "\" + new URL(\".\", import.meta.url) + \"", // relative configuration for assets (TBD with UI5 Web Components team)
				})
				]
			},
			[...]
			]
		\end{lstlisting}
		
		The same file is used for bundling and each time it is bundled differently. The \texttt{replace} method of to configuration object replaces a string inside the input file. In this case the string \textit{placeholder} is first replaced with \textit{0-9-0} and in the second bundle configuration with \textit{1-1-0}. Also the files generated in the process are called differently, the first one is called \texttt{tableViewMixedVersions1.js} and the second \texttt{tableViewMixedVersions2.js}.
		Even though the elements inside those files are actually the same, they are handled as differently, since their tags differ in the browser. Elements register for the first view are called for example \texttt{<ui5-bar-0-9-0>} and for the second then \texttt{<ui5-bar-1-1-0>}.
		
		That means out of the same file, several differently bundled and named files were generated. Each one of the bundled files register the same elements under different tags which are therefore treated as new elements.

\section{Webpack Module Federation with Angular} 
\label{wmf_implementation_prototype}

The three implemented versions for the the WMF landscape are similar to one another, the only difference can be found are the dependencies and their configured sharing. Listing \ref{list:wmf_sameversions_shell} shows the configuration for the \textit{sameVersion} environment.

\begin{lstlisting}[language=JavaScript, caption=Content of \texttt{webpack.config.js} of the shell of the same versions WMF project, label=list:wmf_sameversions_shell, xleftmargin=.01\textwidth, xrightmargin=.01\textwidth]
	[...]
	module.exports = {
		[...]
		plugins: [
		new ModuleFederationPlugin({
			name: "shell",
			filename: "remoteEntry.js",
			shared: share({
				"@angular/core": { 
					singleton: true, 
					strictVersion: false, 
					requiredVersion: '= 12.2.0'
				},
				"@angular/common": { 
					singleton: true, 
					strictVersion: false, 
					requiredVersion: '= 12.2.0' 
				},
				...sharedMappings.getDescriptors()
			})
			
		}),
		sharedMappings.getPlugin()
		]};
\end{lstlisting}

As it can be seen the shared dependencies are defined in between lines 22 and 24. Line 18 and 19 define the name of the application in the landscape and the name of the file after bundling. For the above case it has to be mentioned that the remotes are separate applications in their own runtime. Therefore, they have to be imported via the network, thus no \texttt{remote} property is configured in this file. To add the dynamically loaded remotes, a service had to be developed which imports the remotes at runtime. This service is shown in listing \ref{list:wmf_lookup_service}.\cite{wmf_angular_dynamicfederation}

\begin{lstlisting}[language=JavaScript, caption=Content of \texttt{lookup.service.ts} for remote module loading in shell applications, label=list:wmf_lookup_service,  xleftmargin=.01\textwidth, xrightmargin=.01\textwidth]
	[...]
	@Injectable({ providedIn: 'root' })
	export class LookupService {
		lookup(): Promise<PluginOptions[]> {
			return Promise.resolve([
			{
				remoteEntry: 'https://angular-wmf-same-mfe1.surge.sh/remoteEntry.js',
				remoteName: 'mfe1',
				exposedModule: './Mfe1',
				
				displayName: 'Mfe1',
				componentName: 'Mfe1Component'
			},
			[...]	
			] as PluginOptions[]);
		}
	}
\end{lstlisting}

This service serves the information of the remotely loaded modules to plugins for actual rendering. The rendering itself is done in a proxy plugin component. This component defines a plain template as some sort of placeholder for the remotes. \cite{wmf_angular_dynamicfederation}
\newpage
\begin{lstlisting}[language=JavaScript, caption=Content of \texttt{plugin-proxy.component.ts} for remote module loading in shell applications, label=list:wmf_pluginproxy,  xleftmargin=.01\textwidth, xrightmargin=.01\textwidth]
	[...]	
	@Component({
		selector: 'plugin-proxy',
		template: `
		<ng-container #placeHolder></ng-container>
		`
	})
	export class PluginProxyComponent implements OnChanges {
		@ViewChild('placeHolder', { read: ViewContainerRef, static: true })
		viewContainer: ViewContainerRef;
		
		constructor(
		private injector: Injector,
		private cfr: ComponentFactoryResolver) { }
		
		@Input() options: PluginOptions;
		
		async ngOnChanges() {
			this.viewContainer.clear();
			
			const Component = await loadRemoteModule(this.options)
			.then(m => m[this.options.componentName]);
			
			const factory = this.cfr.resolveComponentFactory(Component);
			const compRef = this.viewContainer.createComponent(factory, null, this.injector);		
		}
	}
\end{lstlisting}

Line 8 of listing \ref{list:wmf_pluginproxy} defines the \texttt{ng-container} with the identifier called \texttt{placeholder}. This identifier is used in the code below to select and actually fill the container with a remote module. The functionality is placed in one of Angulars Lifecycle hook methods \texttt{ngOnChanges}, which is triggered when changes to input properties occur.\cite{wmf_angular_lifecyclehooks} 
In there, the container is first cleared, then a remote loading option is selected from the array of the \texttt{lookup.service.ts}. The following line creates an Angular component out of the loaded remote and inserts it into the placeholder container, using the imported dependencies.
The type of the plugin options was defined in an interface, exporting a type definition. The code for is displayed in listing \ref{list:wmf_plugintype}.
\newpage
\begin{lstlisting}[language=JavaScript, caption=Content of \texttt{plugin.ts} for remote module loading in shell applications, label=list:wmf_plugintype,  xleftmargin=.01\textwidth, xrightmargin=.01\textwidth]
	import { LoadRemoteModuleOptions } from '@angular-architects/module-federation';
	export type PluginOptions = LoadRemoteModuleOptions & {
		displayName: string;
		componentName: string;
	};
\end{lstlisting}

The previously imported \texttt{@angular-architects/module-federation} dependency offers an existing type interface for that use case. This is extended by two more properties in line 4 and 5 of listing \ref{list:wmf_plugintype}.
By configuring the above service and component, it is made possible to load federated modules via the network into the host application.
The configuration for a federated module can be seen in listing \ref{list:wmf_sameversions_mfe1}.

\begin{lstlisting}[language=JavaScript, caption=Content of \texttt{webpack.config.js} of the mfe1 remote app of the same versions WMF project, label=list:wmf_sameversions_mfe1,  xleftmargin=.01\textwidth, xrightmargin=.01\textwidth]
	[...]
	module.exports = {
		[...]
		plugins: [
		new ModuleFederationPlugin({
			name: "mfe1",
			filename: "remoteEntry.js",
			exposes: {
				'./Mfe1': './src/app/mfe1.component.ts'
			},
			shared: share({
				"@angular/core": { 
					singleton: true, 
					strictVersion: false, 
					requiredVersion: '<= 12.2.0' 
				},
				"@angular/common": { 
					singleton: true, 
					strictVersion: false, 
					requiredVersion: '<= 12.2.0' 
				},
				"@fundamental-ngx/core": { 
					singleton: true, 
					strictVersion: false,
					requiredVersion: '0.33.0-rc.214' 
				},
				...sharedMappings.getDescriptors()
			})
		}),
		sharedMappings.getPlugin()
		]};
\end{lstlisting}

As mentioned in chapter \ref{Chapter2}, to share dependencies every participant has to bulk in. Therefore, similarities can be found in the sharing configurations of the remotes and hosts. 
Between line 7 and 11, the actual federation of the module is configured. The reference to the module is later bundled in a file with the name defined in the \texttt{filename} property. In this case it is the \texttt{remoteEntry.js}.
This is the file, which is automatically generated when the remote is compiled and serves as the entry point for the application when it is loaded into the host. Therefore this is the file accessed via the server url in the \texttt{lookup.service.ts} \ref{list:wmf_lookup_service}.
As soon as the script is loaded via the service, the exposed module paths and names become known to the host and can be used to load the module. Thus the \texttt{expose} property contains a Java Script object, which maps the path to the actual component. In this case to \texttt{./Mfe1}.\cite{wmf_concepts}

The implementations of the other versions are similar and only differ in the the shared dependencies configured. The effects of a multi-version environment and how it is handled by the Module Federation were explained in chapter \ref{Chapter2}.

	% \glsresetall
\chapter{Conclusion} % Main chapter title
\label{Chapter7}

\lhead{Chapter 7. \emph{Conclusion}}

The following chapter will conclude and evaluate the results shown in chapter \ref{Chapter6}. First the a direct comparison between similar or equal landscapes and technologies is made, followed by the comparison of different landscapes and technologies.
For the evaluation of the results the introduced KPIs will be used as examples, references or for further calculations.
 
\section{Direct evaluation of the landscapes}

\subsection{Angular and Vue landscapes}

The Angular and Vue landscapes implemented were the following.

\begin{itemize}[noitemsep]
	\item Angular NPM 1
	\item Angular NPM 2
	\item Angular Unpkg 1
	\item Angular Unpkg 2
	\item Vue NPM 1
	\item Vue NPM 2
	\item Vue Unpkg 1
	\item Vue Unpkg 2
\end{itemize}

The naming refers to the used technology to load or access the required resources for the apps. Unpkg, as mentioned before, is a public cloud CDN. The idea behind those implementation was to prove that not only the CDN technology affects the performance of a micro frontend, but also the used frameworks.
Comparing the numbers of those two landscapes it becomes obvious, the Angular landscapes load more resources (counted by the URLs loaded KPI) in a shorter loading time for all implementations. 
Also looking at the initial loading times, calculated via the loading procedure where caching was disabled, for all implementations Angular performed better compared to Vue.

It has to be mentioned, that the actual goal of this implementation was to compare the technology of the CDN, not the used frameworks. It is still mentioned though just to prove that this decision should be considered when planning developing projects for micro frontends. 

Looking the the numbers of the NPM implementations and the ones with Unpkg in use, following aspects come to show.

\begin{enumerate}
	\item The amount of requested URLs is significantly increased
	\item Even though approximately ten times the amount of URLs is requested by the landscape, the initial loading times decreases by ca. 60 MS for the Angular Unpkg landscapes and for Vue it's even more.
	\item The cache usage of the landscapse significantly increases when a CDN is used. This statement is counted by the amount of \textit{none established} ID occurrences. For each connection not established, a resource is loaded from the cache/disk. Thus the faster loading times, when cache is enabled
	\item A similar result is present in the loaded bytes from connection type. Compared to the NPM implementation, the Unpkg environments load significantly more bytes from \textit{disk} or \textit{memory}. This behavior was anticipated in the CDN implementations, since it is one of the desired features of this technologies. Since every resource has a designated URL from where it is loaded from, the browser can distinguish if it already loaded or not.
	\item Additionally the average loaded content size of the Unpkg apps is lower, since only specific resources can be requested from the CDN. Therefore big bundles with unused features are generally not present there.
	\item The graph \ref{fig:unsed_imported_1} shows a similar result. Even though the below graph \ref{fig:unsed_imported_2} shows, that approximately half of the imported bytes were not used according to the Lighthouse report, the absolute amount of loaded bytes is significantly lower, compared with the NPM environments.
	\item When comparing the variances of the loading times, the Unpkg implementation show a lower value. Reasons for that could be either the protocol used by the CDN server or the smaller resource sizes of the loaded resources
\end{enumerate}

In comparing the KPIs \textbf{loading time in MS, resource sizes in bytes and amount of cached resources} the Unpkg implementation follow the expected pattern. Additionally looking at the variance the Unpkg environments show a less variant loading time for all applications compared with the NPM implementations. 
One behavior was not expected though, initially it was assumed that the initial loading time of a CDN landscape would be significantly higher. Reason for that was the assumed effect of the network latency, since the resources are loaded from a remote server, instead from an integrated bundle inside the project itself. Nonetheless, in case of the prototype this behavior could not be confirmed, since even the initial loading times for the Unpkg apps, were lower compared with the NPM apps. It is assumed that, reason for that result are the efficiently picked resources. Instead of importing whole bundles of libraries only necessary components or resources were added as import in the Unpkg landscape. Thus the loaded byte size of those are so low.

Mentioning the \textit{"efficiently picked resources"} another KPI which should be considered for all landscapes, is the effort connected to using a corresponding technology. This metric is hard to measure though, since it highly influenced by the individual using the technology. A developer whom is familiar with Webpack for instance, would have less trouble using and implementing the Module Federation. Therefore this metric is not easily quantifiable. Still, in the context of this thesis, the author will try and provide a subjective opinion on his experiences with the implementations he has done as generally as possible. 
In case of the CDN the effort of implementation is comparably low. From a developers point of view it's even less effort, since no libraries have to be maintained in a central resource or package manager file (namely \texttt{package.json}). Nonetheless it has to considered what type of CDN is used. Is a self hosted or not? 
In case of the prototype implementation, it was a public cloud CDN which already had all the required resources available. For a self hosted CDN, this might now always be the case. Also when deciding to host an own CDN maintenance, development and deployment costs have to be put into account. This part was explained in chapter \ref{Chapter2}.
In summary, the CDN technology is a comparably easy way to avoid redundancies in a micro frontend landscape. It still is connected to certain obstacles, when ones use case is highly specific and requires certain customizations on CDN side. 

Also multi-version landscapes are not supported by a CDN. That means that redundancies still can occur if the same resources is imported under different version tags. This use case is not directly covered by a CDN. If the resource itself, has some sort of scoping feature a support can be provided (e.g. Custom Scopes by SAPUI5 Web Components). But this is not part of the CDN. Other technologies offer more support on that part.

\subsection{Compound and WMF landscapes}

The functionality of the Web Component and Module Federation landscapes to avoid redundant libraries were explained the chapter \ref{Chapter4} and \ref{Chapter5}. Thus this section will focus on the direct comparison of these two environments. Since those two landscapes also included components to simulate heterogenic, multi-version micro frontends, this has to be considered in the comparison. 
Starting with values introduced in chapter \ref{Chapter6}, the first thing to attract attention should be the difference in the average loading time of the landscapes. For the caching disabled scenarios, the WMF landscape requires almost a fifth of the time, the Compound landscape requires. With caching disabled the difference is only a third but still significant. Additionally the bytes loaded by the landscapes differ. WMF loads almost double the amount of bytes compared to the Compound environment in a shorter period on average. Additionally WMF does this with barely using the cache, even with caching enabled. Where the Compound environment loaded approximately 756055 bytes on average from memory, the WMF only imported 1959 bytes on average from the cache. Therefore, based in this comparison the Webpack Module Federation seems to be on top.

Still further aspects have to be considered when using those technologies. First is the multi-version handling. Web Components offer means to handle different versions of a component which is to be registered. Since in case of the prototype a component library (namely SAPUI5 Web Components) this feature was present in it. Still a self developed component can implement a similar functionality. This again would be connected to more effort developing Web Components for a Compound landscape in Luigi, but it is not entirely impossible.
When using an existing component library, this feature might already be present as it is the case for the SAPUI5 Web Components. 
Since each other version of a component is registered under a different tag, with a version suffix attached to it, the result might lead to redundancies again. That means the a component called e.g. \texttt{ui5-table} would be registered and imported twice into the same landscape under different names. For instance version 1 would be \texttt{ui5-table-v1} and version 2 \texttt{ui5-table-v2}. Thus even though the different versions are handled in a distinguishable way, the redundancies would increase.

The Module Federation offers different means for solving this issue. By sharing certain dependencies and with a definition of a required version, redundancies are not entirely eliminated in a multi version WMF landscape, but handled more elegantly. During the data collection of this landscape, a phenomenon appeared which indeed appears to intended by the Federations developers. Shared dependencies like for instance \texttt{@angular-common - v1.1.2} which are shared over the landscape, are always loaded via the network even if caching is enabled, but never in full size. On initial loaded this dependency would have approximately 1.2 MB on size, but when reloading the page with cache enabled, this dependency is loaded again via the net but with 412 KB in size.
This behavior appeared for all shared dependency in the WMF landscape. This behavior is enabled via the Module Federation itself, since the shared dependencies are lazy loaded as chunks. \cite{wmf_the_good_and_ugly}

Again it seems to be that the WMF is superior on that regard compared to Web Components. Still following aspects are not considered yet, the effort of using the technology. As mentioned in the previous section this is no actual KPI but rather a subjective opinion of the author. When directly compared, the effort of implementing or developing micro frontends, was by far higher when using WMF compared to Web Components. Even taking into account that a component library was used, still the Module Federation required more expertise with the Webpack bundler.
The possibilities with WMF are more vast, but to one unfamiliar with the necessary bundler Webpack, kept in secret. The documentation offers good hints and explanations for options and syntax inside the configuration, but it is documented in a more general way and if the developer wants to use it in combination with a Webpack based framework like Angular or Vue, the whole operation becomes more experimental. In the case of Angular, a separate dependency is required to publish the hidden \texttt{webpack.config.js} via which the Module Federation is enabled in the first place. And this configuration has to be done for every module or remote, federated to the landscape. Thus, assuming a real life scenario with independent isolated teams, working on the same micro frontend landscape using different UI frameworks for their remote modules, this might become a bigger obstacle, as it was in case of the prototype. Not only the bundle size of the heterogenic landscape would increase since every shared dependency has to be bundled and published in the landscape, but the the routing inside the landscape becomes a challenge itself. In case of Angular for instance, the inner routers do not recognize route changes and one router has to import another applications router manually to be able to communicate route changes.\cite{wmf_the_good_and_ugly}

Web Components on the other hand are standard. The result and assigning process is the same. Different UI frameworks offer means to register developed components as Web Components (e.g. Angular Elements). Thus the developer can work framework independent within a familiar environment and expect the same result as another developer working with tools of his choice.
Additionally Web Components are not bound to certain technology stacks, not like the WMF which requires the Webpack bundler to enable it. Ergo Web Components have less limitations and more stability due to their standardization.

In summary, both technologies offer means to solve the issue of avoiding redundant libraries, but when it comes to handle multi version landscapes, WMF offers more elegant ways compared to Web Components. On the other hand using the WMF limits the developers to a certain technology stack and is not always easy or effortless implemented depending on the UI framework in use. Also certain obstacles are present in the WMF as of now and require workarounds to solve them.\cite{wmf_the_good_and_ugly} On that regard Web Components offer easier means and ways for implementation due to their standardized aspects.

\section{Final recommendations}

The previous section contain general summaries for each respective technology, based on either empiric data, official sources or subjective experience. None the low the below listing will provide final recommendations when and how best to use the introduced and implemented technologies based on the conclusions made in this transcript.

\begin{itemize}
	\item \textbf{Content Delivery Networks} - CDN offer an easy way to reduce redundancies in micro frontend landscapes by centralizing the landscapes resources to one point. From developer perspective only the way of importing the resources changes and no further changes are applied. Therefore, this is the easiest and fast way of achieving the goal of avoiding redundant libraries on micro frontend landscapes. Nonetheless, what it offers in simplicity it lacks in flexibility. Multi version support is not always present and can not be entirely solved by this technology. Also a public CDN might not always have the necessary resources required by the landscape and hosting an own CDN is connected to high maintenance and developing costs (depending on its size). Therefore, if the use case describes a homogenic micro frontend landscape one without different versions, CDNs is the way to go. If different versions of the same resource are required, then there are more elegant ways than this technology.
	
	\item \textbf{Web Components} - As a web standard it is a save way of again avoid redundancies, maybe not in libraries but rather in the used components them self. Providing reusable components to the browser which are not affected by the isolation of micro services. Additionally existing Web Component libraries offer means to scope different versions of its components, providing a way to handle multi versions inside the landscape. With its standardized aspect it also does not limit the developer to certain technology stacks and is compatible with most common UI frameworks. Some even offer framework features to create Web Components out of their components like Angular Elements. If the use case requires a lot of reusable components with as less redundancies as possible, Web Components offer the best way to provide that service.
	
	\item \textbf{Webpack Module Federation} - A rather new technology compared to the other two, promises to excel where the other two lack. Via federalizing remotes, which again can be any piece of precompiled code, and embedding it into a host application, the Module Federation offers a flexible way of sharing, UI components, modules or utility services inside a micro frontend landscape. The possibilities are vast using this technology. But this offer comes for a price, one has to use the Webpack bundler and for certain features or issues workarounds are required. 
	Generally it can be said, this technology if used correctly, is applicable to almost any use case, but this might be connected with some obstacles and workarounds, which could be easier solved with other means. 
\end{itemize}

As a closing word for the recommendation, when the main goal is to avoid redundant libraries in a micro frontend landscape, each of the introduced technologies offers means to do that. But each comes with their own kind of trade off. Also when picking from one of the above choices, one might want to consider side effects and benefits to be gained from that choice. Thus the final recommendation is highly dependent on the use case and requirements for the landscape to be developed.
Additionally it is not excluded or impossible to combine those technologies. 


	% \glsresetall
\chapter{Conclusion} % Main chapter title
\label{Chapter8}

\lhead{Chapter 8. \emph{Conclusion}}

This chapter will conclude and evaluate the results shown in chapter \ref{Chapter7}. For the evaluation of the results, the introduced KPIs will be used, either as examples or references.
Lastly, a final recommendation will be made, under which circumstances or for which use case the used technologies should best be employed.
 
\section{Evaluation of the landscapes}

This section will evaluate the results for each implemented Node and compare it to its respective counterpart. This section is split into two subsections, one for each prototype.

\subsection{NPM/CDN prototype}

The Angular and Vue Nodes implemented were the following:

\begin{itemize}[noitemsep]
	\item Angular NPM 1
	\item Angular NPM 2
	\item Angular Unpkg 1
	\item Angular Unpkg 2
	\item Vue NPM 1
	\item Vue NPM 2
	\item Vue Unpkg 1
	\item Vue Unpkg 2
\end{itemize}

The naming refers to the technology used to load or access the required resources for the apps. 
Unpkg.com, as mentioned before, is a public cloud CDN. 
The idea behind those implementations is to prove that not only the CDN technology affects the performance of a micro frontend, but also the used frameworks.
When, comparing the numbers of those two landscapes, it becomes obvious that the Angular Nodes load more resources (counted by the URLs loaded KPI) in a shorter loading time for all implementations. 
Also, looking at the initial loading times calculated through the loading procedure where caching was disabled, Angular performed better compared to Vue.js for all implementations.

It has to be mentioned that the actual goal of this implementation was to compare the technology of the CDN, not the used frameworks. It still shows, that the framework has an affect on the performance of a micro frontend, though.

The following conclusions can be drawn by comparing the numbers of the NPM implementations and the ones using Unpkg.com:

\begin{enumerate}
	\item The number of requested URLs is significantly increased when a CDN is used.
	
	\item Even though approximately ten times the number of URLs is requested by the CDN landscapes, the initial loading time decreases by $\sim$60 ms for the Angular Unpkg landscapes. For Vue it is even more.
	
	\item The cache usage of the CDN landscapes significantly increases. This information is taken from the number of \textit{none established} ID occurrences. For each connection not established, a resource is loaded from the cache, resulting in the faster loading times when cache is enabled.
	
	\item A similar result is present in the loaded bytes from connection type. Compared to the NPM implementations, the Unpkg environments load significantly more bytes from \textit{disk} or \textit{memory}. This behavior was anticipated in the CDN implementations, since it is one of the desired features of this technology. Every resource has a designated URL it is loaded from, thus the browser can distinguish if it is already present or not.
	
	\item Additionally, the average loaded content size of the Unpkg apps is lower, since only specific resources are requested from the CDN. Therefore, big bundles with unused features are generally not present.
	
	\item Graph \ref{fig:unsed_imported_1} shows that approximately half of the imported bytes were not used according to the Lighthouse report. However, the absolute number of loaded bytes is significantly lower, compared with the NPM environments.
	
	\item When comparing the variances of the loading times, the Unpkg implementations show a lower value. Reasons for that could either be, the proximity of the CDN server, the bandwidth or the smaller sizes of the loaded resources.
\end{enumerate}

In retrospective, the KPIs \textbf{loading time in MS, resource sizes in bytes} and \textbf{amount of cached resources} of the Unpkg.com implementations follow the expected patterns. 
Additionally, looking at the variance, the Unpkg environments show a less variant loading time for all applications compared with the NPM implementations. 
One behavior was not expected though: It was assumed that the initial loading time of a CDN landscape would be significantly higher. 
The reason behind that assumption was the effect of the network latency, since the resources are loaded from a remote server, and not from an integrated bundle inside the project itself. 
Nonetheless, in case of the prototype this behavior could not be confirmed, as even the initial loading times for the Unpkg apps were lower compared to the NPM apps. 
This could be due to the efficiently picked resources. 
Instead of importing whole bundles of libraries, only necessary components or resources were added as imports in the Unpkg Nodes, thus the loaded byte size of those are so low.

On the topic of \textit{"efficiently picked resources"}, another KPI which should be considered for all landscapes, is the effort connected to using a corresponding technology. 
This metric is hard to measure, though, since it is highly influenced by the individual using the technology. 
A developer who is familiar with Webpack, for instance, would have less trouble using the Module Federation. 
Therefore, this metric is not easily quantifiable. 
Still, in the context of this thesis, the author will try and provide a subjective opinion based on his experiences with the implementations he has done as generally as possible. 

In case of the CDN, the effort of implementation was comparably low. 
From a developer's point of view it is even less effort, since no libraries have to be maintained in a central resource or package manager file (namely \texttt{package.json}). 
Nonetheless, it has to be considered what type of CDN is used.
In case of the prototype implementation, it was a public cloud CDN which already had all the required resources available. 
For a self-hosted CDN, this might not always be the case. 
Also when deciding to host a CDN maintenance, development and deployment costs have to be put into account. 
This was explained in chapter \ref{Chapter2}.
In summary, the CDN technology is a comparably easy way to avoid redundancies in a micro frontend landscape.
It is still connected to certain obstacles, when the use case is highly specific, and requires certain customizations on CDN side. 
Also, multi-version landscapes are not supported by a CDN. 
That means that redundancies still can occur if the same resource is imported under different version tags. 
This use case is not directly covered by a CDN. 
If the resource itself has some sort of scoping feature, a support can be provided (e.g. Custom Scopes by UI5 Web Components), but this is not part of the CDN. 
The other technologies offer more support on that part, as described in the next section.

\subsection{Web Component/WMF landscapes}

The functionality of the WC and WMF landscapes to avoid redundant libraries was explained in chapters \ref{Chapter4} and \ref{Chapter5}. 
Therefore, this section will focus on the direct comparison of these two environments. 
Since these two landscapes also include the aspect of heterogenic, multi-version micro frontends, this fact is considered in the comparison. 
Starting with values introduced in chapter \ref{Chapter6}, the first thing to attract attention should be the difference in the average loading times of the landscapes. 
For the caching disabled scenarios, the WMF landscape requires almost a fifth of the time compared to the compound landscapes. 
With caching enabled, the difference is only a third, but still significant. 
Additionally, the bytes loaded by the landscapes differ. 
WMF loads almost twice the number of bytes compared to the WC environments in a shorter period on average and it does that barely using the cache, when it is enabled. 
Whereas the WC environment loaded approximately 756055 bytes on average from memory, the WMF only imported 1959 bytes on average from the cache. 
Therefore, based in this comparison, the WMF seems to come out on top.

Still, further aspects have to be considered when using those technologies. 
The first is the multi-version handling. 
WCs offer means to register different versions of a component, e.g. by adding a suffix to its tag name. 
In case of the prototype, the UI5 Web Components library provided such a feature. 
A self-developed component can implement a similar functionality. 
This would again be connected to more effort developing Web Components for a compound landscape in Luigi, but it is not entirely impossible.
When using an existing component library, this feature might already be present, as it is the case for the UI5 Web Components. 
Since each other version of a component is registered under a different tag, with a version suffix attached to it, the result might lead to redundancies again. 
That means a component called e.g. \texttt{ui5-table} would be registered and imported twice into the same landscape under different names. 
Thus, even though the different versions are handled in a distinguishable way, the redundancies would increase.

The Module Federation offers a different feature for solving this issue. 
By sharing certain dependencies and with a definition of a required version, redundancies are not entirely eliminated in a multi-version WMF landscape, but handled more elegantly. 
During the data collection of this landscape, a phenomenon appeared which seems to be intended by the Module Federation's developers. 
Shared dependencies, like for instance \texttt{@angular-common - v1.1.2}, are always loaded via the network, even if caching is enabled, but never in full size. 
On initial loading, this dependency would be approximately 1.2 MB in size, but when reloading the page with cache enabled, it is loaded again but with 412 KB in size.
This behavior appeared for all shared dependencies in the WMF landscape. It is enabled via the Module Federation itself, since the shared dependencies are lazy loaded as chunks.\cite{wmf_the_good_and_ugly}

Again it seems that the WMF is superior in that regard compared to WCs. 
Still, one aspect has not been considered yet: The effort of using the technologies. 
As mentioned in the previous section, this is no actual KPI but rather a subjective opinion of the author. 
When compared directly, the effort of implementing or developing micro frontends was higher by far when using WMF compared to Web Components. 
Even taking into account that a component library was used, the Module Federation still required more expertise with the Webpack bundler.
The possibilities with WMF are versatile, but difficult to use by a layman unfamiliar with the necessary bundler. 
The documentation offers good hints and explanations for options and syntax inside the configuration, but it is documented in a more general way, and if the developer wants to use it in combination with a Webpack based framework like Angular or Vue.js, the whole operation becomes more experimental. 
In the case of Angular, a separate dependency is required to publish the hidden \texttt{webpack.config.js} via which the Module Federation is enabled in the first place. 
And this configuration has to be done for every module or remote, federated in the landscape. 
Applied to a real life scenario, this might become a bigger obstacle, as it was in case of the prototype.
Independent, isolated teams might work on the same micro frontend landscape, using different UI frameworks for their remote modules. 
Not only would the bundle sizes of the heterogenic landscape increase, since every shared dependency has to be bundled and published in the landscape, but the routing inside the landscape might become a challenge itself. 
In case of Angular, the inner routers do not recognize route changes. One router would have to import another application's router manually, in order to be able to communicate route changes.\cite{wmf_the_good_and_ugly}

Web Components, on the other hand, are based on standards. 
The result and assigning process are the same. 
Different UI frameworks provide features to register developed components as Web Components (e.g. Angular Elements). 
Thus, the developer can work framework-independent within a familiar environment and expect the same result as another developer working with the tools of their choice.
Additionally, Web Components are not bound to certain technology stacks - unlike the WMF which requires the Webpack bundler to enable it. 
Therefore, Web Components have less limitations and more stability due to their standardization.

In summary, both technologies can solve the issue of avoiding redundant libraries, but when it comes to handle multi-version landscapes, WMF offers more elegant ways compared to Web Components. 
On the other hand, using the WMF limits the developer to a certain technology stack and is not always easy or effortless implemented depending on the UI framework in use. 
Also, as of now certain obstacles are present in the WMF which require workarounds \cite{wmf_the_good_and_ugly}. 
On that regard, Web Components offer easier ways for implementation due to their standardized aspects.

\section{Final recommendations}

The previous sections contain general summaries for each respective technology, based on either empirical data, official sources or subjective experience. 
The following descriptions will provide final recommendations when and how best to use the introduced and implemented technologies, based on the conclusions made in this transcript.

\begin{description}
	\item[Content Delivery Networks:] CDNs offer an easy way to reduce redundancies in micro frontend landscapes by centralizing the landscape's resources to one point. 
	From a developer-perspective, only the method of importing the resources changes. 
	Therefore, this is the easiest and fastest way of achieving the goal to avoid redundant libraries in micro frontend landscapes. 
	Nonetheless, what it offers in simplicity, it lacks in flexibility. 
	Multi-version support is not always present and can not be entirely solved by this technology. 
	Also, a public CDN might not always have the necessary resources required by the landscape and hosting a CDN is connected to high maintenance and developing costs (depending on its size). 
	Therefore, if the use case describes a homogenic micro frontend landscape, one without different versions, CDN is the way to go. 
	If different versions of the same resource are required, there are more elegant ways than this technology.
	
	\item[Web Components:] As a web standard, it is a safe way to avoid redundancies - maybe not in libraries, but rather in the used components themselves. 
	By providing reusable components to the browser which are not affected by the isolation of micro frontends, this technology reduces the amount of used components in the landscape. 
	Additionally, existing WC libraries provide features to scope different versions of its components to handle multiple versions inside the landscape. 
	With its standardized aspect, it also does not limit the developer to certain technology stacks and is compatible with most common UI frameworks. 
	Some of these even offer features to create WCs from their projects like Angular Elements. 
	If the use case requires a lot of reusable components with as few redundancies as possible, WCs offer the best way to provide that service.
	
	\item[Webpack Module Federation:] A rather new technology compared to the other two, WMF promises to excel in where the other two are lacking. 
	It can federalize any piece of pre-compiled code and serve it to the landscape. 
	This module or remote is then embedded into a host application. 
	The remote itself can either be a UI component, a module or a utility service, thus providing a flexible way of sharing code inside its landscapes. 
	The possibilities of using this technology are vast. 
	But, this offer comes for a price: One has to use the Webpack bundler and for certain features or issues, workarounds are required. 
	Subsequently, it can be said that this technology, if used correctly, is applicable to almost any use case, as long as it involves a homogenic landscape\footnote{This was explained in section \ref{wmf_implementation_prototype} of chapter \ref{Chapter6}.}.
\end{description}

As a closing word for the recommendation: 
When the main goal is to avoid redundant libraries in a micro frontend landscape, each of the introduced technologies offers means to do that. 
But each comes with its own trade-off. 
Also, when picking one of the above choices, side effects and benefits have to be considered. 
Therefore, the final recommendation is highly dependent on the use case and requirements for the landscape to be developed.


	
	
	\newpage
	%----------------------------------------------------------------------------------------
	%	BIBLIOGRAPHY
	%----------------------------------------------------------------------------------------
	
	\label{Sources}
	\lhead{}
	\begingroup
	\raggedright
	\addcontentsline{toc}{chapter}{Bibliography}
	\bibliographystyle{plain} % Use the "unsrtnat" BibTeX style for formatting the Bibliography
	\bibliography{sources} % The references (bibliography) information are stored in the file named "Bibliography.bib"
	\endgroup
	
	%----------------------------------------------------------------------------------------
	%	THESIS CONTENT - APPENDICES
	%----------------------------------------------------------------------------------------
	
	
	\listoffigures % Write out the List of Figures
	
	%\listoftables % Write out the List of Tables
	
	%----------------------------------------------------------------------------------------
	%	ABBREVIATIONS
	%----------------------------------------------------------------------------------------
	
	\clearpage % Start a new page
	
	\setstretch{1.5} % Set the line spacing to 1.5, this makes the following tables easier to read
	
	\lhead{\emph{Abbreviations}} % Set the left side page header to "Abbreviations"
	\listofsymbols{ll} % Include a list of Abbreviations (a table of two columns)
	{
		\textbf{API} & \textbf{A}pplication \textbf{P}rogramming \textbf{I}nterface \\
	}
	
	\appendix % Cue to tell LaTeX that the following 'chapters' are Appendices
	
	% Include the appendices of the thesis as separate files from the Appendices folder
	% Uncomment the lines as you write the Appendices
	
	% Appendix Template

\chapter{Table of contents for the archive} % Main appendix title

\label{appendix1} % Change X to a consecutive letter; for referencing this appendix elsewhere, use \ref{AppendixX}

\lhead{Table of contents for the archive} 

\section{Folder Data}

Contains all data collected via the data collection process of chapter \ref{Chapter7}.

\subsection{analysisExcel.xlsm}

Central excel containing all calculation, graphs and data gathered via the testing process. 

\subsection{centralJson.csv}

Aggregated \texttt{.json} file, of all Lighthouse reports. 

\subsection{createAllCsv.sh}

Shell script, used to run a cascading process for the shell scripts in the sub-folders. 
Via this script, the generation of \textbf{analysisExcel.xlsm} and \textbf{centralJson.csv} is triggered.
 
\subsection{Folder angular-npm-data}

Folder containing all the data for the Angular NPM landscapes. 
The data is available as \texttt{.har}, \texttt{.json} as well as \textbf{.csv} files. 
The files themselves are split into, whether the data was collected with cache enabled or disabled.
Additionally, aggregated files for that landscapes are present, as well as a shell script to trigger the creations of every file in the folder.

\subsection{Folder angular-unpkg-data}

Folder containing all the data for the Angular Unpkg landscapes. 
The data is available as \texttt{.har}, \texttt{.json} as well as \textbf{.csv} files. 
The files themselves are split into, whether the data was collected with cache enabled or disabled.
Additionally, aggregated files for that landscapes are present, as well as a shell script to trigger the creations of every file in the folder.

\subsection{Folder vue-npm-data}

Folder containing all the data for the Vue NPM landscapes. 
The data is available as \texttt{.har}, \texttt{.json} as well as \textbf{.csv} files. 
The files themselves are split into, whether the data was collected with cache enabled or disabled.
Additionally, aggregated files for that landscapes are present, as well as a shell script to trigger the creations of every file in the folder.

\subsection{Folder vue-unpkg-data}

Folder containing all the data for the Vue Unpkg landscapes. 
The data is available as \texttt{.har}, \texttt{.json} as well as \textbf{.csv} files. 
The files themselves are split into, whether the data was collected with cache enabled or disabled.
Additionally, aggregated files for that landscapes are present, as well as a shell script to trigger the creations of every file in the folder.

\subsection{Folder compound-views-vanilla-data}

Folder containing all the data for the WC landscapes. 
The data is available as \texttt{.har}, \texttt{.json} as well as \textbf{.csv} files. 
The files themselves are split into, whether the data was collected with cache enabled or disabled.
Additionally, aggregated files for that landscapes are present, as well as a script to trigger the creations of every file in the folder.

\subsection{Folder wmf-views-data}

Folder containing all the data for the WMF landscapes. 
The data is available as \texttt{.har}, \texttt{.json} as well as \textbf{.csv} files. 
The files themselves are split into, whether the data was collected with cache enabled or disabled.
Additionally, aggregated files for that landscapes are present, as well as a shell script to trigger the creations of every file in the folder.

\section{Folder Master\_Development}

Containing all the developed prototype projects for this thesis. 

\subsection{Folder master\_thesis\_dev}

Contains the project for the CDN/NPM landscapes and their respective React Core application.

\subsection{Folder master\_thesis\_dev\_compound}

Contains the projects for the WC compounds landscapes.

\subsection{Folder master\_thesis\_dev\_wmf}

Contains the projects for the WMF compounds landscapes.

\subsection{Folder master\_thesis\_express\_app}

An express app, was developed in the early stages of the thesis and later abandoned, since a local data source was picked.

\section{Folder Self-developed-Scripts}

Contains all self-developed scripts for generating the \texttt{.csv} from the \texttt{.har} and \texttt{.json} files.

\subsection{Folder centralJson2Csv}

Generates the centralJson.csv file in the archive. It is a JavaScript script, which adds a command to the console, which is later executed via the shell scripts in the Data directory.

\subsection{Folder har2csv}

Generates a single \texttt{.csv} file based on single \texttt{.har} file. It is a JavaScript script, which adds a command to the console, which is later executed via the shell scripts in the Data directories.

\subsection{Folder hars2csvs}

Generates a single \texttt{.csv} file based on all \texttt{.har} files in the given directory. It is a JavaScript script, which adds a command to the console, which is later executed via the shell script in the Data directories to generate the central \texttt{.csv} files for each data directory.

\subsection{Folder json2Csv}

Generates a single \texttt{.csv} file based on single \texttt{.json} file. It is a JavaScript script, which adds a command to the console, which is later executed via the shell scripts in the Data directories.

\subsection{Folder jsons2Csvs}

Generates a single \texttt{.csv} file based on all \texttt{.json} files in the given directory. It is a JavaScript script, which adds a command to the console, which is later executed via the shell script in the Data directories to generate the central \texttt{.csv} files for each data directory.


	% Appendix Template

\chapter{Appendix 2} % Main appendix title

\label{appendix2} % Change X to a consecutive letter; for referencing this appendix elsewhere, use \ref{AppendixX}

\lhead{Appendix 2} % Change X to a consecutive letter; this is for the header on each page - perhaps a shortened title

	% Appendix Template

\chapter{Appendix 3} % Main appendix title

\label{appendix3} % Change X to a consecutive letter; for referencing this appendix elsewhere, use \ref{AppendixX}

\lhead{Appendix 3} % Change X to a consecutive letter; this is for the header on each page - perhaps a shortened title
	
	\addtocontents{toc}{\vspace{2em}} % Add a gap in the Contents, for aesthetics
	
\end{document}
